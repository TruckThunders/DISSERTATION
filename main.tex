%% Thesis file for NMSU Astronomy Dept. Thesises...Thesi...Theses...ya know that thing you exchange for a degree.
% Last edited May 2012 by J. Coughlin. % jlcough@nmsu.edu
% Requires the following packages (which should be default in your latex installation): latexsym, graphicx, amssym, natbib, verbatim, caption.
% If you use \includegraphics then you will also need graphics, and if you use epsfig you will also need epsfit, and likely epstopdf for pdflatex.
% In testing there was some issues with mac users and the caption package, so I have included the needed caption*.sty files with this distro just in case to override the system default.

\documentclass[12pt,preprint]{aastex-thesis}  % Using a custom-modified version of aastex v5.2. Need to use preprint style. (You can do preprint2 for double-column if you want for reading over - the bulk of thesis will come out okay in 2 columns but some pages won't, like the header pages and ToC).  Default is 12pt for NMSU thesis. Note this means that /normalsize is 12pt, /footnotesize is 10pt, /small is 11pt, /scriptsize is 8 pt. Having footnotes 10pt complies with NMSU thesis guidelines.

\usepackage{epsfig}  % Allows you to use epsfig, of course.
\usepackage{epstopdf}  % Converts eps files to pdf on the fly - have to have to work with pdflatex.
%\usepackage{endfloat}  % If you uncomment this line, it moves all the figures to the end of the paper. Useful if you want to print thesis without figures in the way for proofreading or whatever
\usepackage{graphics}  % Need if using /includegraphics, but not if using all /epsfig. Doesn't hurt to include though.
\usepackage{natbib}  % Gotta have for bibtex of course.
\usepackage{iondefs} % Definitions of ion symbols if you work with ions a lot...isn't it little ionic? don't ya think? just a little ionic. Like a thousand tables, when all you need is a figure...
\usepackage{listings}  % Package to directly include source code in the document. E.x.  \lstinputlisting{MyProg.cpp}

\begin{document}

\doublespace

\pagenumbering{roman}  % Makes it so all the intro pages are numbered i, ii, iii, iv, etc.

\pagestyle{empty}  % We don't want a page number (i) printed on the first page

\begin{center}
UNDERSTANDING THE CIRCUMGALACTIC MEDIUM THROUGH ARCHIVAL SPECTROSCOPIC STUDIES\\
\bigskip
BY\\
\bigskip
NIGEL LAWRENCE MATHES, B.S., M.S.
\end{center}
\vspace{1.0in}
\begin{center}
A dissertation submitted to the Graduate School\\
\bigskip
in partial fulfillment of the requirements\\
\bigskip
for the degree\\
\bigskip
Doctor of Philosophy
\end{center}
\bigskip
\begin{center}
Major Subject: Astronomy
\end{center}
\vspace{1.0in}
\begin{center}
New Mexico State University\\
\bigskip
Las Cruces New Mexico\\
\bigskip
July 2017
\end{center}
  % Title Page - title.tex

\pagestyle{plain} % Now start showing page numbers

\noindent "Understanding the Circumgalactic Medium Through Archival Spectroscopic Studies," a dissertation prepared by Nigel L. Mathes in partial fulfillment of the requirements for the degree, Doctor of Philosophy, has been approved and accepted by the following:

\singlespace

\bigskip
\begin{flushleft}
\hrulefill
\newline
Louí-Vicente Reyes, PhD
\newline
Dean of the Graduate School

\vspace{0.5in}

\hrulefill
\newline
Christipher W. Churchill
\newline
Chair of the Examining Committee

\vspace{0.5in}

\hrulefill
\newline
Date
\vspace{0.5in}
\newline
Committee in charge:
\end{flushleft}

\doublespace

\indent Dr. Christopher W. Churchill, Chair\\
\indent Dr. Rene A.M. Walterbos\\
\indent Dr. Anatoly A. Klypin\\
\indent Dr. Michael Engelhardt\\
  % Grad School Approval Page - approval.tex

\begin{center}
DEDICATION
\end{center}

I dedicate this work to my cat, Stellar, for enduring my relentless inane nicknames and providing love and affection in the form of headbutts. 
  % Dedication Page - dedica.tex

\begin{center}
ACKNOWLEDGMENTS
\end{center}

My greatest thanks to Stellar, for also making sure I wake up each morning by biting my nose.

To the NMSU Spacegrant Consortium, for grant in aid awards in 2013 and 2014.

To Sigma Xi for awarding a travel grant making possible collaboration with Jane Charton at The Pennsylvania State University.

To the European Southern Observatory Science Archive Facility and the Keck Observatory Archive, for organizing and providing access to nearly two decades of observatios.

To Dr. Michael T. Murphy for his aid in collaboration, both providing access to spectacular data and for his detailed input into the science presented in this dissertation.

To Dr. John O'Meara for providing access to yet another spectacular data set.

To Dr. Jane Charlton for collaboration and mentoring well beyond my tenure as her student at Penn State.

To all those who have supported quasar absorption line studies.
  % Acknolwedgements Page - ackno.tex

\begin{center}
 VITA
\end{center}

\medskip

\begin{center}
 EDUCATION
\end{center}


\begin{flushleft}
\begin{tabular}{ll}
2011-2013 &  M.S., Astronomy\\
 & New Mexico State University\\
 & Las  Cruces, New Mexico, USA\\
2007-2011 &  B.S., Astronomy \& Astrophysics\\
 & The Pennsylvania State University\\
 & State College, Pennsylvania, USA
\end{tabular}
\end{flushleft}

\medskip

\begin{center}
AWARDS AND GRANTS
\end{center}
\begin{flushleft}
\begin{tabular}{ll}
2013,2014 & New Mexico Space Grant Consortium Graduate Fellowship\\
2012 & Sigma Xi Research Grant
\end{tabular}
\end{flushleft}

\medskip

\begin{center}
PROFESSIONAL ORGANIZATIONS
\end{center}
\begin{flushleft}
\begin{tabular}{l}
American Astronomical Society
\end{tabular}
\end{flushleft}

\medskip

\singlespace

\begin{center}
PUBLICATIONS
\end{center}

\hangindent=0.5in \noindent \underline{Mathes, N.L.}, Churchill C.W., and Murphy, M.T., 2017, The Astrophysical Journal, NUMBERS. \textit{The Vulture Survey I: Analyzing the Evolution of MgII Absorbers}

\hangindent=0.5in \noindent Muzahid, S., Kacprzak, G.G., Churchill, C.W., Charlton, J.C., Nielsen, N.M., \underline{Mathes, N.L.}, and Trujillo-Gomez, S., 2015, The Astrophysical Journal, 811, 132. \textit{An Extreme Metallicity, Large-scale Outflow from a Star-forming Galaxy at z  ̃ 0.4}

\hangindent=0.5in \noindent \underline{Mathes, N.L.}, Churchill C.W., Kacprzak, G.G., Nielsen, N.M., Trujilo-Gomez, S., Charlton, J., Muzahid, S., The Astrophysical Journal, 792, 128. \textit{Halo Mass Dependence of H I and O VI Absorption: Evidence for Differential Kinematics}

%\medskip
%\begin{center}
%CONFERENCE PROCEEDINGS
%\end{center}

%\hangindent=0.5in \noindent \underline{Coughlin, J. L.}, L\'opez-Morales, M., Harrison, T. E., Ule, N., Hoffman, D. I. 2011, in ASP Conf. Ser. 448, 16$^{th}$ Cambridge Workshop on Cool Stars, Stellar Systems, and the Sun, ed. C. Johns-Krull (Seattle, WA:ASP), 121. \textit{New Low-Mass Eclipsing Binaries from Kepler}

\medskip

\begin{center}
FIELD OF STUDY
\end{center}
\begin{flushleft}
Major Field: Galaxy Evolution\\
Minor Field: Quasar Absorption Line Spectroscopy
\end{flushleft}
  % Vita page - vita.tex

\doublespace

\begin{center}
ABSTRACT
\end{center}

\bigskip

\begin{center}
The Vulture Survey: Feasting on the Bones of Archival Spectra Left to Die\\
\bigskip
BY\\
\bigskip
NIGEL L. MATHES, B.S., M.S.
\end{center}

\bigskip

\begin{center}
Doctor of Philosophy\\
New Mexico State University\\
Las Cruces, New Mexico, 2012\\
Dr. Christopher W. Churchill, Chair
\end{center}

\bigskip
%\bigskip

% ABSTRACT MAY NOT EXCEED 350 WORDS

We present detailed measurements of the redshift path density, equivalent width distribution, column density distribution, and redshift evolution of ${\MgII}$ and ${\CIV}$ absorbers as measured in archival spectra from the UVES spectrograph at the Very Large Telescope (VLT/UVES) and the HIRES spectrograph at the Keck Telescope (Keck/HIRES) to equivalent width detection limits below $0.01$~{\AA}. This survey examines 432 VLT/UVES spectra from the UVES SQUAD collaboration and 170 Keck/HIRES spectra from the KODIAQ group, representing 580 unique sightlines, allowing for detections of intervening ${\MgII}$ absorbers spanning redshifts $0.1 < z < 2.6$ and intervening ${\CIV}$ absorbers spanning redshifts $1 < z < 5$. We employ an accurate, automated approach to line detection which consistently detects redshifted absorption doublets. We find that an increased number of high column density ${\MgII}$ and ${\CIV}$ absorbers at $z = 2$ drives an overall increase in the quantity of metals around galaxies as compared to the present epoch. We conclude that galaxies eject more metal enriched gas into their halos around $z = 2$ than at any other redshift through star formation driven outflows. We determine that weak ${\MgII}$ and ${\CIV}$ absorbers, those with equivalent widths less than $0.3$~{\AA}, are physically distinct and evolve separately from very strong absorbers, which have equivalent widths greater than $1.0$~{\AA}. Over this same time period, evolving ionizing conditions in the halos of galaxies gives rise to an increasing population of low equivalent width, passive ${\MgII}$ and ${\CIV}$ absorbers. From $z = 2$ to the present, feedback processes decline and we observe fewer very strong systems. From $z = 2$ to $z = 5$, SOMETHING ELSE HAPPENS. EXPLAIN THIS SOMETHING ELSE.
  % Abstract page - abstract.tex

\singlespace

\tableofcontents  % Command to generate the table of contents
\newpage

\listoftables  % Generate the list of tables
\addcontentsline{toc}{section}{LIST OF TABLES}  % Makes it so the table of contents includes an entry for the list of tables and the page number its on
\newpage
\addcontentsline{toc}{section}{LIST OF FIGURES}  % Ditto as above but for the list of figures.
\listoffigures  % Generate the list of figures

%\addcontentsline{toc}{section}{DATA ON COMPACT DISC}

\begin{center}
DATA ON COMPACT DISC
\end{center}
% Must include operating system; software used to create the information; and any other information another researcher might need to access the data
The FITS files were created on a MacBookPro2,2 with a 2.16 GHz Intel Core 2 Duo and 4 GB of RAM running Mac OS Leopard (v10.5.8). The C code was compiled using the GNU project's gcc (v4.2).

\begin{tabbing}
DIRTY/\= \\
\>dirtydust\_v2rev2\_8mar2010.tar.gz\\
\>dirtyv2\_r88\_20sep10.tar.gz\\
arbitrary  geometry code/ \\
\>day90.00\_myshell\_grid30\_tau1.00\_ff0.75\_pos.fits \\
\>day90.00\_myshell\_grid30\_tau1.00\_ff0.75\_tau\_per\_pc.fits \\
\>day90.00\_myshell\_subgrids\_grid30\_tau1.00\_ff0.75\_pos.fits \\
\>day90.00\_myshell\_subgrids\_grid30\_tau1.00\_ff0.75\_tau\_per\_pc.fits \\
\>day90.00\_mytorus\_theta5.00\_grid30\_tau1.00\_pos.fits \\
\>day90.00\_mytorus\_theta5.00\_grid30\_tau1.00\_solid\_pos.fits \\
\>day90.00\_mytorus\_theta5.00\_grid30\_tau1.00\_solid\_tau\_per\_pc.fits \\
\>day90.00\_mytorus\_theta5.00\_grid30\_tau1.00\_tau\_per\_pc.fits \\
\>makefits \\
\>writingFITS\_USEME.c \\
\end{tabbing}
  % Page if you have a CD included with your thesis. If you do not have a CD, comment out this line

\addcontentsline{toc}{section}{LIST OF ABBREVIATIONS}  % Comment this out if you do not want/have a list of abbreviations

\begin{center}
LIST OF ABBREVIATIONS
\end{center}

\begin{tabular}{rl}
AGN       &     Active Galactic Nucleus\\
FITS      &     Flexible Image Transport System\\
FWHM      &     Full-Width Half Max\\
HIRES     &     The high-resolution optical spectrograph of the Keck I Telescope\\
ISM       &     Interstellar Medium\\
MW        &     Milky Way\\
NASA      &     National Aeronautics and Space Administration\\
NSF       &     National Science Foundation\\
STScI     &     Space Telescope Science Institute\\
UV        &     Ultraviolet\\
UVES      &     The high-resolution optical spectrograph of the VLT\\
VLT       &     Very Large Telescope\\
\end{tabular}
  % Page that contains a list of abbreviations. This is optional, so if you don't have many or don't want it, just comment out this line

\doublespace

\clearpage
\resetcounters  % Custom command in aastex-thesis to re-set counters so figures, tables, and equations are labeled 1.1, 1.2, 1.3, 2.1, 2.2, etc., instead of 1.1, 1.2, 1.3, 2.4, 2.5

\pagenumbering{arabic}  % Make page numbers 1,2,3... starting with the Introduction

\section{\MakeUppercase{Introduction}}
\label{ch1}

The field of extragalactic astronomy has blossommed in the past decade, with the advent of large telescopes, grand uniform surveys, and the computing power to simulate complex processes on universal scales. By combining observational constraints and cosmological hydrodynamic simulations, we have managed to create galaxies that look like those we have observed for decades. However, as we make more detailed observations revealing the specifics of ever-increasingly complex physical processes, we find that the models do not truly mimic reality. In the end, progress can be made through more detailed observations and more complex simulations.

We know so much more about the material inside galaxies than we do about the material outside galaxies. There exist large reservoirs of gaseous material outside the visible extent of galaxies, in their halos, which must be studied in detail in order to understand how galaxies grow and evolve. In addition, as our understanding of all forms of matter evolves, so must our analysis methods. Individual case studies still have their merit; however, due to the sheer volume of our observable universe and the wild variance in all properties of observable structures, detailed surveys are the next step forward in understanding exactly how our universe works.

This work attempts to begin that process by surveying in great detail the wealth of archived observations at the world's largest, most productive observatories. These data sets make no concessions in resolution or detection sensitivity, and hope to characterize the properties of matter in galaxy halos in great detail so as to trace and predict the full spectrum of gaseous structures throught the universe.

\subsection{Galaxy Formation and Evolution}
\label{ch1:galaxyevolution}

One of the most elusive yet important goals in extragalactic research aims to discover how galaxies form, grow, and evolve in an effort to truly understand our place in the universe. The creation of a galaxy begins with a cosmic overdensity, where matter congregates to begin the process of building complex structures of stars, gas, and dust. Feedback from star formation and AGN, along with mergers, shape the galaxy and its surroundings into complex environments. The stage of most uncertainty throughout this process involves feedback, the injection of energy into the gaseous medium within and outside of galaxies which, as has been shown in simulations, is required to produce realistic looking galaxies.

The Milky Way serves as the closest and best example of a galaxy. However, galaxies come in all kinds of different shapes and sizes. To truly understand how we came to be, as humans on the planet Earth, we must work to understand how our galaxy came to be, and what processes played part in creating the local environment we live in today.

\subsection{The Circumgalactic Medium}
\label{ch1:cgm}

Characterizing the baryonic gas processes within and surrounding galaxies is central to understanding their formation and evolution. Quantifying the spatial extent, kinematics, and, in particular, the recycling and/or escape fraction of circumgalactic gas are of primary importance in that they place direct observational constraints on simulations of galaxies and provide insights into the workings of galaxy evolution.

In general, the gas structures in and around galaxies can be divided into three broad categories: the interstellar medium (ISM), the circumgalactic medium (CGM), and the intergalactic medium (IGM). The CGM, being the gas reservoir that interfaces with the star-forming ISM, outflowing stellar-driven winds, and the accreting IGM, may contain up to 50\% of the baryonic mass bound to galaxies \citep{Tumlinson2011} and account for up to 50\% of the baryons unaccounted for in galaxy dark matter halos \citep{Werk2014}.  As such, the CGM may play the most critical role in governing the properties of galaxies \citep[e.g.,][]{Oppenheimer2010, MAGIICAT3}, including regulatory physics leading to the observed stellar mass function \citep[e.g.,][]{Behroozi2013} and the stellar mass-ISM metallicity relationship \citep[e.g.,][]{Tremonti2004}.

The physical extent of the CGM and the transition zone between the CGM and IGM are currently open questions. Studies by \citet{Steidel2010}, \citet{Prochaska2011}, and \citet{Rudie2013} indicate a transition from the CGM to the IGM at $\log N({\HI}) \simeq 14$ and a projected distance of $\sim\! 300$~kpc from galaxies at $z\sim 2.5$.  At this redshift, $\log N({\HI}) \simeq 14$ corresponds to an overdensity of $\log \rho_{\hbox{\tiny H}}/\bar{\rho}_{\hbox{\tiny H}} \simeq 0.5$, whereas at $z\simeq 0$, this overdensity would suggest a CGM/IGM transition at $\log N({\HI}) \simeq 13$ \citep[see][]{Dave1999}.  Indeed, \citet{Ford2013mass} show that, in the over-dense regions hosting galaxies, the extent of the {\HI} at fixed column density increases with virial mass, suggesting that a single fiducial physical size for the CGM may not apply across the entire mass range of galaxies; it may be more appropriate to scale CGM properties relative to the virial radius \citep[e.g.,][]{CWC2013Masses,MAGIICAT3}.

%The cgm is pretty big. Like, bigger than most things you know about.

\subsection{The Baryon Cycle}
\label{ch1:baryoncycle}

One of the most important questions in modern studies of galactic evolution asks, how do baryons cycle into and out of galaxies, and how does this cycle determine the growth and evolution of galaxies themselves? More specifically, how does the process of gas accretion, star formation, and subsequent supernovae-driven feedback shape both the galaxies themselves and their circumgalactic medium (CGM)? By using spectroscopic observations of quasars, we can identify and analyze metal line absorbers in and around the halos of foreground galaxies. Though absorption line studies by themselves cannot directly answer these questions, the statistical results from such studies can provide vital information from which further progress can be made.

%Baryons go in, more baryons come out. YOU can't explain that.

\subsection{Quasar Absorption Line Spectroscopy}
\label{ch1:absorptionlines}

The details of this complex process should be observable in the relationship between the stars in the galaxy and the gas phase material located between the stars within the galaxy (interstellar medium; ISM), just outside the galaxy (circumgalactic medium; CGM), and between galaxies themselves (intergalactic medium; IGM). Unfortunately, detailed observations of this gas are difficult as it remains mostly neutral (CITATION), and therefore does not strongly emit. Therefore, massive efforts have been undertaken to observe this gas in absorption using the most advanced telescopes.

A major problem in studying galaxy evolution at high redshifts is that faint, low mass galaxies cannot be observed with current facilities. These smaller, more numerous galaxies pose major problems when attempting to characterize the ionization conditions of the early universe and detailing the growth history of galaxies. They are all but invisible due to their low luminosities, which makes it nearly impossible to completely inventory galaxy populations and find modern day galaxy analogs at high redshift. Studying properties of galaxy evolution, then, must involve a luminosity-independent tracer of the processes which grow and affect galaxies across cosmic time. One such technique involves observing absorbing gas in quasar spectra, effectively measuring properties of the shadows of gaseous structures around galaxies.

Quasars are exceptionally bright objects, often found at large distances, or high redshift. Their luminosities can exceed SOMETHING BIG, and they can be observed at distances corresponding to light travel times near to that of the beginning of the universe. They serve as extreme cosmic lighthouses, illuminating material located between observers and the quasars themselves.

By taking a spectrum of a quasar, we can learn about material at large distances which may not emit light of its own. By absorbing the light of the quasar at specific wavelengths, this mater, or gas, imprints onto the spectrum a characteristic absorption feature. These absorption features can be measured in order to determine their redshift, and also their underlying physical properties.

\subsection{A Brief History of {\MgII} Absorption Surveys}
\label{chp1:mgiibackground}

One of the most prolific absorption features, the {\MgIIdblt} doublet, traces cool ($T \simeq 10^4~\mathrm{K}$; \cite{Churchill2003}) metal enriched gas in the disks and halos of galaxies. It is one of the best tracers of this gas because it can exist in a wide range of ionizing conditions, ranging in ionization parameter from $-5 < \log U < 1$~\citep{Churchill1999}, it is observable in optical wavelengths for redshifts between $0.1 < z < 2.6$, and it has predictable line characteristics defined by its resonant doublet nature which make it ideal for automated searches.

The origin of ${\MgII}$ absorbing gas is still debated. As summarized in~\cite{Kacprzak2011} and~\cite{Matejek2013}, two separate interpretations exist to explain the origin of strong, high equivalent width ($W_r$) {\MgII} absorbers ($W_r^{\lambda2796} > 0.3$~{\AA}) and weak, low equivalent width ($W_r^{\lambda2796} < 0.3$~{\AA}) ${\MgII}$ absorbers.

For the strong, higher equivalent width systems, multiple correlations exist between the rest frame ${\MgII}$ equivalent width around galaxies and the host galaxy's star formation properties. \cite{Zibetti2007}, \cite{Lundgren2009}, \cite{Noterdaeme2010}, \cite{Bordoloi2011}, and \cite{Nestor2011} all found a correlation betwen $W_r^{\lambda2796}$ and blue host galaxy color, showing that galaxies with more active star formation have more metal enriched gas in their halos. \cite{Bordoloi2014} also found that ${\MgII}$ equivalent width increases with increasing star formation rate density. In addition, spectroscopic observations of star forming galaxies have revealed strong ${\MgII}$ absorption blueshifted $300 - 1000$~{\kms} relative to the host galaxy~\citep{Tremonti2007,Weiner2009,Martin2009,Rubin2010}.

Multple correlations have also been found between the equivalent width of strong ${\MgII}$ absorbers and host galaxy mass. \cite{Bouche2006} found an anti-correlation between galaxy halo mass, derived from the cross-correlation between ${\MgII}$ absorption systems and luminous red galaxies, and ${\MgII}$ equivalent width, showing that individual clouds of a ${\MgII}$ system are not virialized in the halos of galaxies. They interpreted their results as a strong indication that high equivalent width absorbers with $W_r^{\lambda2796} \gtrsim 2$~{\AA} arise in galactic outflows. Marginal anti-correlations between ${\MgII}$ equivalent width and galaxy halo mass using the same cross-correlation method were also reported by~\cite{Gauthier2009} and~\cite{Lundgren2009}. It is imporant to note, however, that~\cite{Churchill2013letter} and~\cite{MAGIICAT3} find no correlation between $W_r^{\lambda2796}$ and halo mass when halo mass is derived from abundance matching. They instead find that galaxies inhabiting more massive dark matter halos have stronger absorption at a given distance.

For the weak, lower equivalent width systems, it seems none of the above correlations hold. \cite{Chen2010b}, \cite{Kacprzak2011}, and \cite{Lovegrove2011} found little evidence for a correlation between galaxy color and ${\MgII}$ equivalent width when restricting their samples to weak absorbers. \cite{Kacprzak2011} measured the orientation of galaxies relative to ${\MgII}$ detections in the sight lines of background quasars and identified low metallicity, low equivalent width ${\MgII}$ absorbers co-planar with some some galaxy disks, implying structures associated with accreting filaments as opposed to outflows, which are more often observed perpendicular to the galaxy disk~\citep{Bordoloi2011,Kacprzak2012-PA,Bouche2012}. Finally, the simulations of~\cite{Stewart2011} and~\cite{Ford2013mass} revealed a reservoir of low-ionization, metal enriched, co-rotating gas around massive galaxies. Together, these studies imply that weak ${\MgII}$ absorption systems may preferentially trace low metallicity infall and co-rotating gas in the circumgalactic medium.

\cite{MAGIICAT1} constructed a sample of ${\MgII}$ absorbers and their associated galaxies and examined both strong and weak ${\MgII}$ absorbers from $0.07 \le z \le 1.1$. In the subsequent analysis of their sample,~\cite{MAGIICAT2} found a more extended ${\MgII}$ absorbing CGM around higher luminosity, bluer, higher redshift galaxies. In addition, in~\cite{MAGIICAT4}, they found that bluer galaxies replenish their ${\MgII}$ absorbing CGM through outflows, whereas red galaxies do not. Finally, in~\cite{MAGIICAT5}, it is made clear that the largest velocity dispersions in ${\MgII}$ absorbing systems are measured around blue, face-on galaxies probed along their minor axis, strongly suggesting that these ${\MgII}$ absorbers originate in bi-conical outflows.

Many surveys have been undertaken to inventory ${\MgII}$ absorbers and examine their evolution. The earliest studies~\citep{Lanzetta1987,Tytler1987,Sargent1988,Steidel1992} found that ${\MgII}$ systems with rest equivalent widths above $0.3$~{\AA} show no evolution in $d\mathcal{N}\!/dz$ between redshifts $0.2 < z < 2.15$. These studies also found that the equivalent width distribution function, $f(W_r^{\lambda2796})$, could be fit equally well with either an exponential or a power-law function. It remains uncertain whether the cosmic distribution of ${\MgII}$ in galactic halos exhibits a fractal, self-similar nature, or if $f(W_r^{\lambda2796})$ flattens at equivalent widths below $W_r^{\lambda2796} < 0.3$~{\AA}.

${\MgII}$ absorption surveys have taken one of two different approaches to try to analyze the global distribution of ${\MgII}$ absorbing gas across cosmic time. \cite{Churchill1999} and \cite{Narayanan2007} aimed to determine more precisely how $d\mathcal{N}\!/dz$ and $f(W_r^{\lambda2796})$ evolve with redshift by surveying weak ${\MgII}$ absorbers. They found that, for these low equivalent width absorbers, $d\mathcal{N}\!/dz$ increases as a function of increasing redshift up until $z = 1.4$. At higher redshifts, $d\mathcal{N}\!/dz$ falls to lower values, though the uncertainties are large. In addition, they found the equivalent width distribution function for weak absorbers is best fit by a power-law, strongly disfavoring an exponential fit to the overall distribution.

The most recent studies have employed new multi-object spectrographs such as the Sloan Digital Sky Survey (SDSS) and the FIRE spectrograph on the Magellan Baade Telescope~\citep{Nestor2005,Matejek2012,Chen2016}. \cite{Nestor2005}, who examined over 1300 intervening ${\MgII}$ absorbers in SDSS quasar spectra with $W_r^{\lambda2796} > 0.3$~{\AA}, found that the equivalent width distribution function is well fit by an exponential. They did not find evidence for redshift evolution in systems with $0.4 < W_r^{\lambda2796} < 2$~{\AA}, but observed an enhancement in the number of $W_r^{\lambda2796} > 2$~{\AA} absorbers per comoving redshift path length as a function of increasing redshift from $z \sim 0$ up to $z \sim 2$. \cite{Matejek2012} and~\cite{Chen2016}, analyzing 279 ${\MgII}$ absorbing systems from $2 < z < 7$ in infrared FIRE spectra, also found that the equivalent width distribution function is well fit by an exponential. They also observed that systems with $W_r^{\lambda2796} < 1.0$~{\AA} show no evolution with redshift, but higher equivalent width systems grow in number density from low redshift to $z \sim 3$, after which the number density declines. Collectively, these surveys imply physical changes in the astrophysical processes or in the state of the gas structures in the environments giving rise to ${\MgII}$ absorption as the universe ages.

% "EW THAT'S MADE OF DEAD TREES"

\subsection{A Brief History of {\CIV} Absorption Surveys}
\label{chp1:civbackground}

The first study to characterize the distribution of intervening CIV absorbers was~/cite{Young1982}, which determined that these lines were randomly distributed in redshift in a manner consistent with absorption from intervening galaxies. /cite{Sargent1988} also showed that the number density of systems per unit redshift decreases with increasing redshift from $1.8 /le z_{em} /le 3.56$. /cite{Steidel1990}, examining strong CIV absorbers at redshifts between $1.3 /le z_{abs} /le 4.0$, found also that the number of CIV absorption systems per unit redshift range decreases with increasing redshift, in a manner inconsistent with a constant comoving density of absorbers. Therefore, they stated, the properties of CIV absorbers must evolve over time.

In terms of kinematics,~/cite{Steidel1990} noted that the peak in the CIV two-point correlation function on velocity scales $200 /le \Delta v /le 600~{\kms}$ appears to have the same power at low and high redshift. ~/cite{Rauch1996} examined the velocity structure of intervening CIV absorbers in the spectra of 3 intermediate redshift quasars and detrmined that gas temperatures were likely near $3.8 /times 10^4~/mathrm{k}$, and the TPCF of CIV systems suggests that there is more than one source of velocity dispersion. They interpret the shape of the resulting TPCF as owing to ensembles of objects with the kinematics of dwarf galaxies on a small scale, while following the Hubble flow on a larger scale. % Reword, as some of these statements are word for word out of the papers they came from.

\cite{Churchill1999} examined data from the Hubble Space Telescope Faint Object Spectrograph for {\CIV} absorbers associated with known {\MgII} absorbers measured in high resolution Keck/HIRES data. They noticed a strong correlation between {\MgII} kinematics and {\CIV} equivalent width. They interpreted that this correlation could imply a connection between outflowing {\MgII} clouds and higher ionization state halo gas.

\cite{Rauch2001} measured the kinematics and column density differences of {\CIV} absorption systems along the lines of sight to three gravitationally lensed quasars. They found the spatial distribution of the gas to be mostly featureless with detectable velocity shear up to $\sim70~\mathrm{kms}$. They found the clouds quiescent, in that the energy transmitted to the gas as measured by the amount of turbulence derived along the line of sight, and therefore determined that the absorbing structures are not internal to galaxies. They posited that {\CIV} absorbers could be gas expelled recently or pre-enriched from earlier star formation.

\cite{Chen2001} examined 14 galaxy absorber pairs, along with 36 galaxies without associated {\CIV} absorption lines. They found that {\CIV} absorption line systems cluster strongly on velocity scales of $v \lesssim 250~\mathrm{\kms}$ and impact parameter scales of $\rho \lesssim 100~h^{-1}~\mathrm{kpc}$. In addition, they note that galaxies of all morphological types and luminosities can possess extended gaseous envelopes, with covering factors near unity, at impact parameters less than $100~\mathrm{kpc}$. They concluded that accreting satellites are the most likely sources of this metal enriched halo gas.

\cite{Simcoe2004} directly measured the metallicity distribution function for the $z \sim 2.5$ intergalactic medium in the spectra of seven quasars. They found no evidence for a universal metallicity floor, as had been suggested for some scenarios of Population III enrichment in the early universe. In addition, they found no trends in metallicity as a function of IGM density.

\cite{Fox2007} studied 63 damped Lyman-$\alpha$ (DLA) systems and 11 sub-DLAs with associated {\CIV} absorption. They detected {\CIV} clouds moving at velocites in excess of the escape speed, determined by measuring the total line width of the neutral gas profile, in roughly 40\% of their systems. They inferred that these clouds may arise in high ionized outflowing winds, powered by galaxies with star formation rates (SFR) of $\sim 2~M_{\odot}~\mathrm{yr^{-1}}$.

\cite{Shull2014} used {\it HST}/COS spectra of background quasars to measure elemental abundances in the low redshift intergalactic medium. They found that {\CIV} has increased in abundance by a factor of 10 from $z \sim 5$ to present.

\cite{Bordoloi2014} measured {\CIV} absorption in the COS-Dwarfs survey, probing the gaseous halos near 43 low-mass galaxies at $z \ge 0.1$. They detected {\CIV} out to $\sim 100~\mathrm{kpc}$, roughly $0.5 R_{vir}$, from the host galaxies. They also reported a tentative correlation between {\CIV} equivalent width and SFR, and concluded that energy-driven star formation winds must expel into the CGM a comparable mass to that of the carbon found in the stars of these galaxies.

\subsection{Project Goals}
\label{chp1:goals}

The goal of this project is to wrest my PhD from the icy grips of my faculty captors.
  % Intoduction

\clearpage
\resetcounters

\label{ch2}
\section{\MakeUppercase{Data}}

\subsection{Spectral Properties}
\label{ch2:spectra}

Properties of UVES and HIRES and things.

Table of spectroscopic observations of quasars.

\subsection{Reduction and Pre-Analysis}
\label{ch2:reduction}

Describe reduction process for each instrument and describe the processes of higher order continuum fitting and things like that.

\subsection{Redshift Path Length}
\label{ch2:pathlength}

Describe g of w z and show the pretty heat map plot and explain it.
  % Chapter 2

\clearpage
\resetcounters

\section{\MakeUppercase{Automated Line Detection}}
\label{chp3}

\subsection{Motivating Automated Line Detection}
\label{ch3:motivation}

Large spectral data sets are becomming the norm in both galactic and extragalactic astronomy. In the Milky Way, the spectra of hundreds of thousands of stars may be analyzed by routines which fit templates to find the best agreement. Unfortunately, when dealing with redshifted systems composed of a multitude of components (stars, gas, AGN), the number of free parameters increases significantly and templates/matching techniques become unfeasable. Therefore, we require a more general, but no less robust line finding algorithm to handle large spectral libraries.

\subsection{Defining the Search Window}
\label{ch3:window}

In any large survey, a well-defined search window is important. In the case of analyzing quasar spectra, we are constrained by several factors: the wavelength coverage of the spectrum, the redshift of the observed quasar, and the confidence with which we can identify individual doublets in regions of low signal-to-noise and regions with atmospheric absorption features (telluric lines).

We first limit the search range to regions of the spectrum redward of the {\Lya} emission, as {\Lya} forest contamination would render automatic detection of weaker metal lines nearly impossible. We also do not search $5000$~{\kms} blueward of the quasar emission redshift in order to avoid absorbers associated with the quasar itself. Finally, we exclude regions of strong telluric absorption bands, specifically from $6277 - 6318$~{\AA}, $6868 - 6932$~{\AA}, $7594 - 7700$~{\AA}, and $9300 - 9630$~{\AA}, because we found that the molecular line separations and ratios can lead to numerous false positives when searching for ${\MgII}$ doublets.

\subsection{Matched Filtering Algorithm}
\label{ch3:detection}

We employ a matched filter search method similar to that of Zhu \& Menard, 2013, seeking line candidates above a certain signal-to-noise ratio threshold. This process involves correlating a template filter (in this case, a tophat function) with a signal to detect the presence of features matching the template. In other words, we convolve the quasar spectrum with a conjugated, reversed filter representing an absorption trough. This method can maximize the signal-to-noise ratio in the presense of stochastic noise.

The filter we choose is a top hat function centered at the wavelength of the desired redshifted absorption line. Its width is selected to match the resolution of the spectrum, which is a function of the slit width used during the exposure. A large variety of slit-widths were used to achieve different resolutions for varying science drivers, but, characteristically for a 1.0 arcsec slit, $R \sim 40,000$ for VLT/UVES and $R \sim 45,000$ for Keck/HIRES. We convolve the filter with the normalized spectrum to generate a normalized power spectrum in redshift space, with absorption features having positive power.

The error spectrum in both instruments is complex, irregular, and has frequent single-pixel spikes which makes uniform normalization impossible. Therefore, we cannot convolve the filter with the error spectrum to derive normalized noise estimates, as is often done in matched filter analysis. Instead, we examine the noise in the derived power spectrum.

To derive the noise, we first divide the power spectrum into smaller chunks which correspond to roughly 3000 pixels in the observed spectrum. We then sigma-clip these chunks to remove absorption features, leaving only the continuum power spectrum. Next, we calculate the standard deviation of this continuum. Finally, we use the standard deviation as the noise to calculate the $S\,/N$ of the absorption features in the power spectrum as the ratio of the normalized power ($S$) to the normalized noise ($N$).

A flagged absorption feature has $S\,/N > 5$. A confirmed doublet detetection for {\MgIIdblt} requires detection of $S\,/N^{\lambda2796} > 5$ and $S\,/N^{\lambda2803} > 3$. Similarly, a detection for {\CIVdblt}, we require $S\,/N^{\lambda1548} > 5$ and $S\,/N^{\lambda1551} > 3$. In addition, our automated routines remove detections with non-physical doublet ratios in unsaturated regions; specifically, we exclude cases where $W_r^{\lambda2803} > W_r^{\lambda2796}$, or $W_r^{\lambda2803} < \left(0.3 \times W_r^{\lambda2796}\right)$. The same convention is also applied to the {\CIV} doublets. The latter constraint is conservative for unsaturated absorbers, as the secondary transition in the doublet is rarely observed less than $0.5 \times$ the primary. We relax this constraint in saturated features. This system could potentially exclude detections where either the primary or secondary line in the doublet is blended with another transition but does not saturate; however, confirmation of these cases requires extra verification from separate ionic transitions, such as ${\FeII}$, which are weaker and not always covered in the spectrum.

All absorption features are visually verified upon completion of the detection algorithm. Multiple feature detections within $\pm 500$~{\kms} of each other are grouped together to generate absorption systems, designated as a single absorber, to be analyzed. Once absorption systems are identified, we calculate the optical depth-weighted median absorption redshift to define the center of the entire absorption system. The formal derivation of this redshift is described in the appendix of~\cite{Churchill2001}.

We also derive an equivalent width detection limit across the spectrum. To do so, we insert modelled Gaussian absorption features across the spectrum and assume a full-width at half maximum (FWHM) defined by the resolution of the instrument to represent unresolved lines. We then solve for the height of the Gaussian, defined as the value at the curve's peak, required to detect the unresolved line with our matched filtering technique at a $S\,/N = 5$. Finally, we integrate to find the equivalent width, and take that value as the minimum detectable equivalent width at a given wavelength. The detection algorithm is therefore self-monitoring. This full equivalent width detection limit spectrum also allows us to accurately characterize the completeness of our sample, along with the full redshift path length searched.

\subsection{The Step-By-Step Recipe for Finding Redshifted Lines}
\label{ch3:steps}

\begin{enumerate}
\item Define Search Window
\item Convert wavelength search window to a redshift range for each ion to search
\item Define redshift resolution based upon the spectrograph resolution
\item Define filter sized based upon FWHM of an unresolved line at the given spectrograph resolution
\item Do the cross-correlation
\begin{itemize}
\item At each redshift, place the filter for each transition ({\MgIIdblt} and {\CIVdblt}).
\item Compute the area between the filter and the observed spectrum - This is the power at that redshift. Higher power corresponds to stronger absorption.
\item Step forward, repeat the calculation. Build up a power spectrum which has the observed power for each transition at a given redshift.
\end{itemize}
\item Normalize the power spectrum.
\begin{itemize}
\item The raw power spectrum's amplitude is tied directly to the wavelength range (or redshift range depending on how you want to think about it at this point), as we are taking the area beneath a variable-length object (in this case, the spectrum or filter). We wish to analyze a power spectrum which ranges between 0 and 1 for any spectrum analyzed. This means we want to subtract the baseline value (the result of the correlation if there is zero absorption detected) from the raw power spectrum, and then divide by the difference between the minimum value (no absorption) and the maximum value (a fully saturated absorption line whose width is equal to or greater than the width of the filter used).
\item This difference ends up being the area of the filter, which, because the spectrum is normalized, is the width of the filter.
\end{itemize}
\item Calculate the normalized error.
\begin{itemize}
\item Split the normalized power spectrum into chunks (HOW BIG ARE THEY).
\item In each of these chunks, sigma clip the power spectrum to eliminate absorption features.
\item Take the standard deviation of the remaining power spectrum. This represents the noise in the continuum after filtering.

\end{itemize}
\end{enumerate}
  % Chapter 3

\clearpage
\resetcounters

\section{\MakeUppercase{Analysis of Absorption Properties}}
\label{ch4}

This is how we look at squiggles and tell you they mean something.

\subsection{Equivalent Width Regions}
\label{ch4:regions}

We put windows on things.

\subsection{Calculating Redshifts and Velocities}
\label{ch4:redshift}

Something something optical depth weighted medians.

\subsection{Equivalent Widths and Kinematic Spreads}
\label{ch4:ew}

Integrate the things and do something with tau.

\subsection{Column Densities}
\label{ch4:column}

AOD stuff here.
  % Chapter 4

\clearpage
\resetcounters

\section{\MakeUppercase{Properties and Evolution of {\MgII} Absorbers}}
\label{ch5}

Now we can finally get down to the science!

\subsection{Number of Absorbers Per Path Length}
\label{ch5:dndx}

%Dat path doe.

The largest sample of quasar spectra originates from the Sloan Digital Sky Survey (SDSS), with more than $10^5$ spectra at present, which employs a spectrograph with an instrumental resolution around $69$~{\kms}, limiting SDSS absorption surveys to strong absorbers, with $W_r^{\lambda2796} \ge 0.3$~{\AA} \citep{Nestor2005,Zhu2013}. Conversely, previous studies of weak absorbers used small samples of quasar spectra, never exceeding 100 quasar spectra \citep{Steidel1992,Narayanan2007,Kacprzak2011}. In this paper, we aim to characterize the evolution of the incidence rate, number of absorbers per redshift path length, comoving line density, and cosmic mass density of all ${\MgII}$ absorbers from redshifts $0.18 < z < 2.57$.

The number of ${\MgII}$ absorbers per redshift path length and its associated variance are defined as

\begin{equation}
\frac{d \mathcal{N}}{d z} = \sum_{i}\frac{1}{\Delta Z_i\,(W_r)},\quad \sigma^2_{\frac{d \mathcal{N}}{d z}} = \sum_{i} \Big[\frac{1}{\Delta Z_i\,(W_r)}\Big]^2,
\label{eqn:dndz}
\end{equation}

\noindent
where we count the number of ${\MgII}$ absorbers, dividing by the total searched redshift path length ($\Delta Z$), defined as

\begin{equation}
\Delta Z_i\,(W_r) = \int_{z_1}^{z_2} g_i\,(W_r, z)\,dz,
\label{eqn:deltaz}
\end{equation}

\noindent
where $g_i(W_r, z)$ is the equivalent width sensitivity function at a given equivalent width detection limit shown in Equation~\ref{eqn:gwz}. The function $g(W_r, z)$, first formulated in \cite{Lanzetta1987}, details the number of spectra in which an absorption feature with a given equivalent width may be detected at the $5\sigma$ level in a given redshift interval.

The comoving ${\MgII}$ line density and its associated variance are defined as

\begin{equation}
\frac{d \mathcal{N}}{d X} = \sum_{i}\frac{1}{\Delta X_i\,(W_r)},\quad \sigma^2_{\frac{d \mathcal{N}}{d X}} = \sum_{i} \Big[\frac{1}{\Delta X_i\,(W_r)}\Big]^2,
\label{eqn:dndx}
\end{equation}

\noindent
where we count the number of ${\MgII}$ absorbers, dividing by the total searched absorption path ($\Delta X$), defined as

\begin{equation}
\Delta X_i\,(W_r) = \int_{z_1}^{z_2} g_i\,(W_r, z) \frac{(1 + z)^2}{\sqrt{\Omega_M (1 + z)^3 + \Omega_{\Lambda}}}\,dz,
\label{eqn:deltax}
\end{equation}

\noindent
where $\Omega_M$ is the cosmic matter density, and $\Omega_{\Lambda}$ is the cosmic density attributed to dark energy. Counting with respect to $\Delta X$ accounts for both cosmological expansion along the line of sight and the transverse separation of objects with unchanging number density and cross section, allowing for more consistent comparisons across redshift.

\begin{figure*}[bth]
\epsscale{1.17}
\plottwo{/Users/Nigel/PROJECTS/VULTURE/PLOTS/hist_z_MgII_dndz.pdf}{/Users/Nigel/PROJECTS/VULTURE/PLOTS/hist_z_MgII_dndx.pdf}
\caption{(a) $d\mathcal{N}\!/dz$ and (b) $d\mathcal{N}\!/dX$ as a function of redshift for different minimum equivalent width thresholds, $W_{r,\mathrm{min}}^{\lambda2796}$. Colors represent different $W_{r,\mathrm{min}}^{\lambda2796}$. The black dotted lines are fits to the distribution of the functional form $f(z) = \frac{c}{H_o} n_0\,\sigma_0\,(1 + z)^{\epsilon}$, with the best fit $\epsilon$ value labelled. We see increasing values of $\epsilon$ with increasing equivalent width, driven by an enhancement of stronger ${\MgII}$ absorbers around redshift 2 compared to lower redshifts. Vertical error bars represent $1\sigma$ uncertainties in each bin.}
\label{fig:dndz_dndx}
\end{figure*}

In Figure~\ref{fig:dndz_dndx}, we plot $d\mathcal{N}\!/dz$ and $d\mathcal{N}\!/dX$, respectively, as a function of redshift for different minimum equivalent width thresholds, such that detected ${\MgII}$ absorbers have equivalent widths greater than $W_{r,\mathrm{min}}^{\lambda2796}$. Error bars in each bin represent $1\sigma$ uncertainties calculated according to Equations~\ref{eqn:dndz} and~\ref{eqn:dndx}. Dotted lines are fit according to the analytical form which allows for redshift evolution in $d\mathcal{N}\!/dX$, defined as,

\begin{equation}
\frac{d \mathcal{N}}{d X}(z) \equiv  \frac{c}{H_o} n(z)\,\sigma(z) =  \frac{c}{H_o} n_0\,\sigma_0\,(1+z)^{\epsilon},
\label{eqn:dndxfit}
\end{equation}

\noindent
where $c$ is the speed of light, $H_o$ is the Hubble Constant, $n_0$ is the comoving number density of ${\MgII}$ absorbers at $z = 0$, $\sigma_0$ is the absorbing cross-section at $z = 0$, and $\epsilon$ is the evolution parameter, defined as the power dependence of $d\mathcal{N}\!/dX$ on redshift. We fit to the binned data using an orthogonal distance regression (ODR) method implemented in Python\footnote{https://docs.scipy.org/doc/scipy/reference/odr.html}. We use this approach in all cases where functional fits are applied, and discuss the robustness of this method in Section~\ref{sec:}. % EDIT NEED SECTION NAME

The product $n_0\,\sigma_0$ represents a comoving opacity of ${\MgII}$-selected absorption line systems by virtue of the units, which are an inverse length, and the analogous absorption coefficient to describe the opacity of material in stellar atmospheres. The full product of $\frac{c}{H_o} n_0\,\sigma_0$ then represents the Hubble optical depth for ${\MgII}$ absorbers. We find that the best-fit value of $\epsilon$ is negative when analyzing the full sample of ${\MgII}$ absorbers, including all detections with measured equivalent widths above $W_r^{\lambda2796} > 0.01$~{\AA}. The evolution parameter, $\epsilon$, then increases with subsequently larger minimum equivalent width thresholds, becoming positive for absorbers with $W_{r,\mathrm{min}}^{\lambda2796} > 1.0$~{\AA}. This trend is driven primarily by an enhancement in $d\mathcal{N}\!/dX$ for the strongest ${\MgII}$ absorbers around $z \sim 2$, relative to lower redshifts. Conversely, at low redshift we observe more weak ${\MgII}$ absorbers per absorption path length. We show in Table~\ref{tab:fitparams} the values of the fit parameters for varying $W_{r,\mathrm{min}}^{\lambda2796}$, along with their $1\sigma$ uncertainties.

\begin{deluxetable}{ccc}
\tablecolumns{3}
\tablewidth{0pt}
\tablecaption{Parameterization of $d\mathcal{N}\!/dX$ \label{tab:fitparams}}
\tablehead{
  %\colhead{(1)} &
  %\colhead{(2)} &
  %\colhead{(3)} \\[2pt]
  \colhead{$W_{r,\mathrm{min}}^{2796}$} &
  \colhead{$\frac{c}{H_o}n_0\,\sigma_0$} &
  \colhead{$\epsilon$} \\[1pt]
  \colhead{[{\AA}]} &
  \colhead{} &
  \colhead{} }

\startdata
0.01 & 2.583 $\pm$ 0.827 & -1.04 $\pm$ 0.38 \\[3pt]
0.30 & 0.446 $\pm$ 0.076 & -0.14 $\pm$ 0.21 \\[3pt]
1.00 & 0.116 $\pm$ 0.043 & 0.31 $\pm$ 0.44 \\[3pt]
2.00 & 0.019 $\pm$ 0.014 & 0.94 $\pm$ 0.85
\enddata
\end{deluxetable}

\begin{figure*}[bth]
\epsscale{1.17}
\plottwo{/Users/Nigel/PROJECTS/VULTURE/PLOTS/hist_nsigma_MgII_dndx.pdf}{/Users/Nigel/PROJECTS/VULTURE/PLOTS/hist_epsilon_MgII_dndx.pdf}
\caption{(a) The comoving number density of absorbers multiplied by the absorbing cross-section, derived by fitting Equation~\ref{eqn:dndxfit} to $d\mathcal{N}\!/dX$, as a function of $W_{r,\mathrm{min}}^{\lambda2796}$ with shaded $1\sigma$ uncertainties. As we examine samples with increasing minimum ${\MgII}$ equivalent width thresholds, either the space density of absorbing cloud structures decreases, the absorbing cross-section decreases, or both parameters decrease. (b) The redshift evolution parameter, $\epsilon$, as a function of $W_{r,\mathrm{min}}^{\lambda2796}$. Weak ${\MgII}$ absorbers are more abundant at low redshift, leading to a negative coefficient $\epsilon$. Absorbers with equivalent widths near $0.3$~{\AA} do not evolve, with $\epsilon \simeq 0$. Strong ${\MgII}$ absorbers evolve away at low redshift, showing a large positive $\epsilon$ increasing towards $z \sim 2$.}
\label{fig:nsigmaepsilon}
\end{figure*}

In Figure~\ref{fig:nsigmaepsilon}, we show the values of $\frac{c}{H_o}n_0\,\sigma_0$ and $\epsilon$ as a function of $W_{r,\mathrm{min}}^{\lambda2796}$. The shaded red areas represent the $1\sigma$ standard deviations derived from the fits to the $d\mathcal{N}\!/dX$ distribution. We show first that the comoving Hubble optical depth of ${\MgII}$ absorbers decreases as a function of $W_r^{\lambda2796}$. This implies that, per unit absorption path length, there are fewer high equivalent width ${\MgII}$ absorbers, and/or that they exist in smaller absorbing structures. We also show that the slope of the redshift dependence, $\epsilon$, increases as a function of increasing $W_{r,\mathrm{min}}^{\lambda2796}$. This evolution parameter, $\epsilon$, changes from negative to positive toward higher equivalent width ${\MgII}$ absorbers, implying that strong ${\MgII}$ absorbers evolve away, decreasing in relative number per absorption path length, from $z = 2$ to present. Conversely, weak ${\MgII}$ absorbers build up over time, increasing in relative number per absorption path length from $z = 2$ to present. We observe no evolution with redshift in absorbers with equivalent widths between $0.3 < W_r^{\lambda2796} < 1$~{\AA}.

We provide a parameterized fit to $\frac{c}{H_o}\,n_0\,\sigma_0$ and $\epsilon$ as a function of $W_{r,\mathrm{min}}^{\lambda2796}$. In Figure~\ref{fig:nsigmaepsilon}(a), we adopt a power-law with a generalized exponential decay to model the $\frac{c}{H_o}\,n_0\,\sigma_0$ distribution, defined as,

\begin{equation}
\frac{c}{H_o}n_0\,\sigma_0\,(\,\psi\,) = \Psi^* (\,\psi\,)^{\alpha} e^{-\,\psi\,^{\beta}} ,
\label{eqn:nsigmafit}
\end{equation}

\noindent where $\psi = W_{r,\mathrm{min}}^{\lambda2796} / W_{r,\mathrm{min}}^*$ to simplify the equation. The best fit parameters are $\Psi^* = 0.24 \pm 0.01$, $W_{r,\mathrm{min}}^* = 1.19 \pm 0.02$, $\alpha = -0.49 \pm 0.01$, and $\beta = 1.50 \pm 0.05$. This parameterization resembles a Schechter function, but we required an exponential drop-off at the high end faster than $e^{-\psi}$, which manifests itself in the form of $\beta$. Next, in Figure~\ref{fig:nsigmaepsilon}(b), we fit a broken power-law to the $\epsilon$ distribution, defined as,

\begin{equation}
%\[
\epsilon\,(W_{r,\mathrm{min}}^{\lambda2796}) =
\left\{
\!
\begin{aligned}
%3.64\,(W_{r,\mathrm{min}}^{\lambda2796})^{0.08}\, -\, 3.44 &\,\, \text{ if }\,\, W_{r,\mathrm{min}}^{\lambda2796} < 1.1~{\angstrom} \\[2mm]
%0.07\,(W_{r,\mathrm{min}}^{\lambda2796})^{3.72}\, +\, 0.15 &\,\, \text{ if }\,\, W_{r,\mathrm{min}}^{\lambda2796} \ge 1.1~{\angstrom},
a_1\,(\,W_{r,\mathrm{min}}^{\lambda2796}\,)^{\gamma_1} + b_1 &\,\, \text{ if }\,\, W_{r,\mathrm{min}}^{\lambda2796} < 1.1~{\AA} \\[2mm]
a_2\,(\,W_{r,\mathrm{min}}^{\lambda2796}\,)^{\gamma_2} + b_2 &\,\, \text{ if }\,\, W_{r,\mathrm{min}}^{\lambda2796} \ge 1.1~{\AA},
\label{eqn:epsilonfit}
\end{aligned}
\right.
%\]
\end{equation}

\noindent where the fit parameters for minimum equivalent width thresholds below $1.1$~{\AA} are $a_1 = 3.13 \pm 1.49$, $\gamma_1 = 0.09 \pm 0.05$, and $b_1 = 2.87 \pm 1.49$. The fit parameters for the power-law with $W_{r,\mathrm{min}}^{\lambda2796} \ge 1.1$~{\AA} are $a_2 = 0.07 \pm 0.01$, $\gamma_2 = 3.73 \pm 0.25$, and $b_2 = -0.21 \pm 0.03$. Combining the fits to $\frac{c}{H_o}\,n_0\,\sigma_0$ and $\epsilon$, we now have an analytic parameterization of $d\mathcal{N}\!/dX$ as a function of $W_{r,\mathrm{min}}^{\lambda2796}$, and $z$, of the form,

\begin{equation}
\frac{d\mathcal{N}}{dX}(W_{r,\mathrm{min}}^{\lambda2796}, z) = \Psi^* \psi^{\alpha} e^{-\psi^{\beta}} (1 + z)^{\epsilon(W_{r,\mathrm{min}}^{\lambda2796})}.
\label{eqn:dndxanalytic}
\end{equation}

\noindent This function can be used in future semi-analytic models to parameterize the physical properties of ${\MgII}$ absorbers in galaxy halos (e.g.~\cite{Shattow2015}).

\subsection{Equivalent Width Distribution}
\label{ch5:ewdistro}

%Schechter that shit.

To calculate the equivalent width frequency distribution $f(W)$, the number of absorbers of a given equivalent width per unit path density, we calculate $d\mathcal{N}\!/dz$ or $d\mathcal{N}\!/dX$ for each equivalent width bin and divide by the bin width. We split the sample into four redshift regimes, ensuring that the number of absorbers in each redshift subsample remains constant. The result is a characteristic number of ${\MgII}$ absorbers per redshift or absorption path length per equivalent width.

\begin{figure*}[bth]
\epsscale{1.17}
\plottwo{/Users/Nigel/PROJECTS/VULTURE/PLOTS/hist_ew_MgII_dndz.pdf}{/Users/Nigel/PROJECTS/VULTURE/PLOTS/hist_ew_MgII_dndx.pdf}
\caption{(a) The equivalent width distribution of ${\MgII}$ absorbers, defined as the redshift path density ($d\mathcal{N}\!/dz$) in each equivalent width bin divided by the bin width. (b) The equivalent width distribution, defined as the comoving line density ($d\mathcal{N}\!/dX$) in each equivalent width bin divided by the bin width. Error bars represent $1\sigma$ uncertainties in each bin. We fit each distribution with a Schechter function, capturing the self-similar power-law behavior of weak ${\MgII}$ absorbers and the exponential power-law cutoff when observing the strongest ${\MgII}$ systems.}
\label{fig:ewdistro}
\end{figure*}

In Figure~\ref{fig:ewdistro}, we plot the equivalent width frequency distribution with respect to either $d\mathcal{N}\!/dz$ or $d\mathcal{N}\!/dX$. We fit each distribution with a Schechter function of the form,

\begin{equation}
\Phi (W_r) = \Phi^* \left(\frac{W_r}{W_r^*}\right)^{\alpha} e^{-W_r / W_r^*} ,
\label{eqn:schechter}
\end{equation}

\noindent where $\Phi^*$ is the normalization, $\alpha$ is the low equivalent width power-law slope, and $W_r^*$ is the turnover point in the distribution where the low equivalent width power-law slope transitions into an exponential cutoff. Table~\ref{tab:schechterew} shows the values of $\Phi^*$, $W_r^*$, and $\alpha$, along with their associated $1\sigma$ uncertainties derived from the fitting routine. This functional fit is motivated by papers such as \cite{Kacprzak2011MgII}, where the authors seek to combine previous surveys of strong ${\MgII}$ absorbers, in which exponential fits were preferred, and surveys of weak ${\MgII}$ absorbers, where power-laws best fit the equivalent width distribution. The power-law nature of the distribution of weak absorbers in our survey is apparent, and the exponential cutoff is motivated by physical limits to the size, density, and velocity widths of ${\MgII}$ absorbing clouds. Examining the distribution as a function of redshift, we find the low equivalent width slope becomes more shallow at $z \sim 2$ compared with the present epoch, with $\alpha = -1.09$ in our subsample with $0.14 \le z < 0.78$ and $\alpha = -0.81$ in our subsample with $1.53 < z \le 2.64$. We observe fewer weak ${\MgII}$ absorbers and more strong ${\MgII}$ absorbers per redshift/comoving absorption path length at $z \sim 2$ than we do at $z \sim 0.5$.

\begin{deluxetable}{lccc}
\tablecolumns{4}
\tablewidth{0pt}
\tablecaption{Schechter Fit to $f(W) = \frac{d\mathcal{N}}{dX} / \Delta W$ \label{tab:schechterew}}
\tablehead{
  %\colhead{(1)} &
  %\colhead{(2)} &
  %\colhead{(3)} \\[2pt]
  \colhead{Redshift Range} &
  \colhead{$\Phi^*$} &
  \colhead{$W^*$} &
  \colhead{$\alpha$} \\[1pt]
  \colhead{} &
  \colhead{} &
  \colhead{$[${\AA}$]$} &
  \colhead{}
}
\startdata
$0.14 - 0.78$  & $0.15 \pm 0.10$ & $1.72 \pm 0.68$ & $-1.09 \pm 0.09$ \\[3pt]
$0.78 - 1.09$  & $0.12 \pm 0.06$ & $2.36 \pm 0.81$ & $-0.99 \pm 0.06$ \\[3pt]
$1.09 - 1.53$  & $0.11 \pm 0.03$ & $2.00 \pm 0.35$ & $-1.02 \pm 0.04$ \\[3pt]
$1.53 - 2.64$  & $0.12 \pm 0.08$ & $2.25 \pm 0.87$ & $-0.81 \pm 0.12$
\enddata
\end{deluxetable}

\subsection{Column Density Distribution}
\label{ch5:columndistro}

%Schechter it HARDER.

To calculate the column density distribution, the number of absorbers of a given column density per unit path density, we calculate $d\mathcal{N}\!/dz$ or $d\mathcal{N}\!/dX$ for each column density bin and divide by the bin width. The result is a characteristic number density of ${\MgII}$ absorbers per redshift or absorption path length as a function of their column densities. It should be noted that at high column densities near $\log (N({\MgII})) = 15~\mathrm{cm^{-2}}$, the measured column densities are lower limits as the AOD method cannot constrain the true column when the absorption line becomes saturated. In these cases, we turn to the {\MgIIsecond} profile, adopting its value of column density instead.

\begin{figure*}[bth]
\epsscale{1.17}
\plottwo{/Users/Nigel/PROJECTS/VULTURE/PLOTS/hist_logN_MgII_dndz.pdf}{/Users/Nigel/PROJECTS/VULTURE/PLOTS/hist_logN_MgII_dndx.pdf}
\caption{(a) The column density distribution of ${\MgII}$ absorbers, defined as the redshift path density ($d\mathcal{N}\!/dz$) in each column density bin dividided by the bin width. (b) The comoving line density ($d\mathcal{N}\!/dX$) in each column density bin divided by the bin width. We fit this distribution with a Schechter function to accurately parameterize the low column density power-law slope and the exponential cutoff and high column densities.}
\label{fig:logndistro}
\end{figure*}

In Figure~\ref{fig:logndistro}, we plot the column density frequency distribution using either $d\mathcal{N}\!/dz$ or $d\mathcal{N}\!/dX$. Again, we fit this distribution with a Schechter function of the same form as Equation~\ref{eqn:schechter}, except with equivalent width replaced with column density. Table~\ref{tab:schechtern} shows the values of $\Phi^*$, $N^*$, and $\alpha$, along with their associated $1\sigma$ uncertainties. We find again that the low column density slope is shallower near $z \sim 2$ than at $z \sim 0.5$. Due to saturation effects, the highest column density measurements may still be lower limits based upon the {\MgIIsecond} profile; therefore, the final high column density bin in $f(N)$ may be regarded as an upper limit. These limits are taken into account in the functional fitting procedures, where using the ODR fitting method we inflate the uncertainties in the positive x and y directions commensurate with the number of limit measurements.

\begin{deluxetable}{lccc}
\tablecolumns{4}
\tablewidth{0pt}
\tablecaption{Schechter Fit to $f(N) = \frac{d\mathcal{N}}{dX} / \Delta N$ \label{tab:schechtern}}
\tablehead{
  %\colhead{(1)} &
  %\colhead{(2)} &
  %\colhead{(3)} \\[2pt]
  \colhead{Redshift Range} &
  \colhead{$\Phi^*$} &
  \colhead{$N^*$} &
  \colhead{$\alpha$} \\[1pt]
  \colhead{} &
  \colhead{$[\times 10^{-16}]$} &
  \colhead{$[\times 10^{14}~\mathrm{cm}^{-2}]$} &
  \colhead{}
}
\startdata
$0.14 - 0.78$ & $8.79 \pm 8.19$ & $1.71 \pm 0.97$ & $-1.14 \pm 0.08$ \\[3pt]
$0.78 - 1.09$ & $6.41 \pm 2.43$ & $2.29 \pm 0.56$ & $-1.10 \pm 0.03$ \\[3pt]
$1.09 - 1.53$ & $4.15 \pm 2.40$ & $2.65 \pm 1.00$ & $-1.12 \pm 0.04$ \\[3pt]
$1.53 - 2.64$ & $7.98 \pm 6.51$ & $2.35 \pm 1.32$ & $-0.91 \pm 0.08$
\enddata
\end{deluxetable}

\subsection{$\Omega_{\MgII}$}
\label{ch5:omega}

%Such cosmology.

We now aim to calculate the matter density of ${\MgII}$ absorbers across cosmic time. To do so, we employ the following customary equation relating the mass density of an ion as a fraction of the critical density today to the first moment of the column density distribution,

\begin{equation}
\Omega_{\hbox{\scriptsize {\MgII}}} = \frac{H_0\  m_{\hbox{\scriptsize {\MgII}}}}{c\ \rho_{c,0}} \int_{N_{min}}^{N_{max}}\, f (N_{\hbox{\scriptsize {\MgII}}})\, N_{\hbox{\scriptsize {\MgII}}}\, dN_{\hbox{\scriptsize {\MgII}}} ,
\label{eqn:omega}
\end{equation}

\noindent where $H_0$ is the Hubble constant today, $m_{\mathrm{Mg}} = 4.035 \times 10^{-23}~\mathrm{g}$, $c$ is the speed of light, $\rho_{c,0}$ is the critical density at present, $f(N_{\hbox{\scriptsize {\MgII}}})$ is the column density distribution of ${\MgII}$ absorbers, and $N_{\hbox{\scriptsize {\MgII}}}$ is the column density. Using our derived fit to the column density distribution, we are able to numerically integrate the first moment from $0 < \log N({\MgII}) < 20~\mathrm{cm^{-2}}$. The upper limit of $\log N({\MgII}) < 20~\mathrm{cm^{-2}}$ is chosen as an effective infinity, where upper integration limits above this value produced no discernable differences in the calculation. Limiting the integral in this manner did nothing more than shorten calculation times when deriving errors, as described below.

The results are shown below in Figure~\ref{fig:omegamgii}. $1\sigma$ uncertainties are derived with a bootstrap Monte-Carlo method. We select random column densities, with replacement, from the sample of measured column densities for all of our ${\MgII}$ absorbers until we reach the sample size. We then recalculate the column density distribution, find the best parameterized Schechter fit, and then integrate and compute Equation~\ref{eqn:omega}. We perfom this task 1499 times to develop a statistical ensemble of values for $\Omega$, with this number of samples representing the underlying scatter in the $\Omega_{\hbox{\scriptsize {\MgII}}}$ distribution at the 99\% confidence level according to~\cite{Davidson2000bootstrap}. We take the standard deviation about the mean of this ensemble of simulated measurements as the $1\sigma$ uncertainty in $\Omega_{MgII}$. We find that the cosmic mass density of ${\MgII}$ increases from $\Omega_{\hbox{\scriptsize {\MgII}}} \simeq 0.8\times10^{-8}$ at $z \sim 0.5$ to $\Omega_{\hbox{\scriptsize {\MgII}}} \simeq 1.3\times10^{-8}$ at $z \sim 2$.

\begin{figure}[bth]
\epsscale{1.2}
\plotone{/Users/Nigel/PROJECTS/VULTURE/PLOTS/Omega_MgII.pdf}
\caption{$\Omega_{\hbox{\scriptsize {\MgII}}}$ as a function of redshift. The cosmic mass density of ${\MgII}$ stays roughly flat near a value of $1 \times 10^{-8}$, with a 0.5 dex increase from $z \sim 0.5$ to $z \sim 2$.}
\label{fig:omegamgii}
\end{figure}

\subsection{Assessing Possible Sightline Bias in the {\MgII} Sample}
\label{ch5:bias}

\begin{figure*}[bth]
\epsscale{1.17}
\plottwo{/Users/Nigel/PROJECTS/VULTURE/PLOTS/hist_ew_MgII_strongz.pdf}{/Users/Nigel/PROJECTS/VULTURE/PLOTS/cdf_fw.pdf}
\caption{(a) The $W_r^{\lambda2796} \ge 0.3$~{\AA} equivalent width frequency distribution for The Vulture Survey in black compared to the exponential fit of~\cite{Nestor2005}, shown as the orange dashed line, and an exponential fit to The Vulture Survey data in purple. The exponential fits are of the form shown in Equation~\ref{eqn:exponential}. (b) The cumulative distribution of The Vulture Survey data and two comparative exponential fits. The P(K-S) value shown compares our survey's data to the fit from~\cite{Nestor2005}. }
\label{fig:bias}
\end{figure*}

The archival quasar spectra used to construct The Vulture Survey were observed for a multitude of reasons. In cases where observations were taken to target a previously known absorption line system, these detections could bias our sample when calculating $d\mathcal{N}\!/dz$ and $d\mathcal{N}\!/dX$. Because early studies which discovered ${\MgII}$ absorbers did not have the sensitivity to detect weak, low equivalent width systems, quasar spectra were never selected based upon the presence of weak absorption. However, some spectra were selected based on the presence of strong, $W_r^{\lambda2796} > 0.3$~{\AA} systems. In order to properly quantify any bias, we compare our work to the large, unbiased sample of the Sloan Digital Sky Survey (SDSS). Specifically, we turn to the work of~\cite{Nestor2005} who inventoried strong ${\MgII}$ absorbers in SDSS quasar spectra. This study employed analysis most similar to our own, identifying absorbers down to $W_r^{\lambda2796} = 0.3$~{\AA}, and produced parametric fits to the equivalent width frequency distribution, allowing for robust comparisons. These authors analyzed the SDSS Early Data Release; however, modern studies have since used larger samples from subsequent SDSS data releases~\citep{Prochter2006,Lundgren2009,Quider2011,Seyffert2013,Zhu2013,Raghunathan2016}. \cite{Zhu2013} specificallty noted that the~\cite{Nestor2005} sample is entirely consistent with their DR7 study.

In~\cite{Nestor2005}, the authors constructed the equivalent width frequency distribution for absorbers with $W_r^{\lambda2796} > 0.3$~{\AA} and found the best fit to this distribution was an exponential function of the form,

\begin{equation}
f(W_r^{\lambda2796}) = \frac{N^*}{W^*} e^{-\,\left(W_r^{\lambda2796} / W^*\right)}
\label{eqn:exponential}
\end{equation}

\noindent where $N^*$ and $W^*$ are constants. In order to compare to the SDSS data, we limit our sample to absorbers with equivalent widths $W_r^{\lambda2796} > 0.3$~{\AA} and calculate the equivalent width frequency distribution. In Figure~\ref{fig:bias}(a), we show $f(W_r^{\lambda2796})$ for The Vulture Survey and an exponential fit to our data, along with the exponential fit of~\cite{Nestor2005}. \cite{Nestor2005} found the best-fit parameters and corresponding $1\sigma$ uncertainties to be $N^* = 1.187 \pm 0.052$ and $W^* = 0.702 \pm 0.017$. When we perform the same analysis, fitting Equation~\ref{eqn:exponential} to The Vulture Survey data, we derive $N^* = 1.39 \pm 0.10$ and $W^* = 0.74 \pm 0.05$. By eye, there appears a slight excess in the number of ${\MgII}$ absorbers above $W_r^{\lambda2796} > 3$~{\AA}, with the frequency of these absorbers roughly $2\sigma$ higher than the~\cite{Nestor2005} exponential fit.

In order to statistically determine the bias in our sample, we perform a Kolmogorov--Smirnov (KS) test to quantitatively measure the similarity between our sample of absorbers with $W_r^{\lambda2796} > 0.3$~{\AA} and the SDSS sample of~\cite{Nestor2005}. We first sample a population of 1331 measured absorber equivalent widths, matching the SDSS sample size, from the exponential fit to the equivalent width ditribution of~\cite{Nestor2005}, incorporating the reported $1\sigma$ scatter in the fit parameters. This allows us to directly compare to the sample of measured ${\MgII}$ equivalent widths from The Vulture Survey.

In Figure~\ref{fig:bias}(b), we show the cumulative distribution of the strong absorbers in The Vulture Survey in black, along with the exponential fit to our data and the fit of~\cite{Nestor2005}. We then perform a two-sample KS test, calculating the P-value, which is the probability that the two samples are drawn from different underlying distributions, defined as P(K-S). To avoid issues due to the random resampling of the~\cite{Nestor2005} distribution, we repeat this exercise one million times as a Monte-Carlo method, generating an ensemble of P(K-S) values. We take the mean P(K-S) value as the true probability that the samples are inconsistent, and the standard deviation about this mean as the $1\sigma$ uncertainty. Our criterion to assert that our sample is consistent with an unbiased sample requires P(K-S)$\, > 0.0027$, which means that it could not be ruled out at the $3\sigma$ level that the two populations are consistent with one another. With P(K-S)$\, = 0.112 \pm 0.049$, and only one instance out of the million Monte-Carlo runs exhibiting a P(K-S) value below 0.0027, we conclude that our sample is consistent with an unbiased sample, even at the $2\sigma$ level, and that the strong absorbers in The Vulture Survey could very well have originated from the same underlying population as the unbiased SDSS quasar spectra sample.

\subsection{Strong vs. Weak Absorbers}
\label{ch5:strongweak}

We have shown a cosmic inventory of ${\MgII}$ absorbing gas from $0.1 < z < 2.6$, measuring $d\mathcal{N}\!/dz$, $d\mathcal{N}\!/dX$, the equivalent width distribution, and the column density distribution down to detection limits as low as $W_r^{\lambda2796} = 0.01$~{\AA}. We aim now to relate the properties of ${\MgII}$ absorbers and their evolution across cosmic time to other known evolutionary processes with the hope of gaining insight into the mechanisms which give rise to ${\MgII}$ absorbing gas.

\cite{Narayanan2007} measured the evolution of weak ${\MgII}$ absorbers from $0.4 < z < 2.4$ in VLT/UVES spectra. They compared to \cite{Churchill1999}, who fitted the equivalent width frequency distribution with a power-law, and to \cite{Nestor2005}, who fitted an exponential to $f(W_r)$. In the case of weak absorbers at $z < 1.4$, \cite{Narayanan2007} found that a power-law with a slope of $\alpha = -1.04$ is a satisfactory fit, confirming the results of \cite{Churchill1999}. When they split their sample into low redshift, with detections between $0.4 < z < 1.4$, and high redshift, with detections between $1.4 < z < 2.4$, they found that the low redshfit sample remained consistent with a power-law but the high redshift sample was best fit by an exponential function. Our data show that a faint end power-law slope of $\alpha = -0.81$ is appropriate for the higher redshift subsample, and we note that this is also consistent with the data of~\cite{Narayanan2007}.

\begin{figure}[bth]
\epsscale{2.25}
\plottwo{/Users/Nigel/PROJECTS/VULTURE/PLOTS/hist_z_MgII_dndz_between.pdf}{/Users/Nigel/PROJECTS/VULTURE/PLOTS/hist_z_MgII_dndx_between.pdf}
\caption{(a) $d\mathcal{N}\!/dz$ as a function of redshift for equivalent widths in the range $0.02 < W_{r}^{\lambda2796} \le 0.3$~{\AA}. We also include the survey data point from $0 < z < 0.3$ of~\cite{Narayanan2005} as shown by the dashed histogram. (b) $d\mathcal{N}\!/dX$ as a function of redshift for the same population. Error bars represent $1\sigma$ uncertainties as calculated in Equations~\ref{eqn:dndz} and~\ref{eqn:dndx}. The red solid lines represent the no-evolution expectation, normalized at $z = 0.9$ and $d\mathcal{N}\!/dz = 1.74$, matching~\cite{Narayanan2007}. The red dot-dashed line in panel (b) represents a linear fit to the binned data of the form $d\mathcal{N}\!/dX = -0.28^{(\pm0.04)}z + 0.92^{(\pm0.04)}$. By this construction, $d\mathcal{N}\!/dX = 0$ at $z = 3.3$, implying we should detect no weak ${\MgII}$ absorbers past this redshift. This equivalent width range comprises exclusively ``weak'' ${\MgII}$ absorbers.}
\label{fig:dndzbetween}
\end{figure}

\cite{Narayanan2007} also analyzed the evolution of $d\mathcal{N}\!/dz$ with redshift for weak ${\MgII}$ absorbers. They found that the distribution follows the ``no evolution'' expectation; that is, the expected number density for a nonevolving population of absorbers in a $\mathrm{\Lambda CDM}$ universe, at redshifts less than $z = 1.5$. At higher redshift, they found that $d\mathcal{N}\!/dz$ for weak absorbers decreases below the no evolution expectation. In Figure~\ref{fig:dndzbetween}(a) we make a direct comparison with \cite{Narayanan2007}, showing $d\mathcal{N}\!/dz$ for $0.02 \le W_r^{\lambda2796} < 0.3$~{\AA} binned in the same manner as their Figure 4. Here, we observe that the apparent peak in $d\mathcal{N}\!/dz$ for weak absorbers occurs near $z = 0.75$, as opposed to $z = 1.2$ in \cite{Narayanan2007}. However, the overall shape of $d\mathcal{N}\!/dz$ for this low equivalent width population remains in good agreement, with the number of weak absorbers per redshift path length rising from $z = 0.4$ to $z \sim 1.0$, following the no evolution expectation, and then falling well below that expectation from $z \sim 1.0$ to $z \sim 2.4$.

In Figure~\ref{fig:dndzbetween}(b), we show $d\mathcal{N}\!/dX$ for weak ${\MgII}$ absorbers. Here, we clearly see evolution in the distribution of low equivalent width absorbers, showing that the comoving line density of weak ${\MgII}$ absorbers steadily decreases as a function of redshift from $z = 0.4$ to $z = 2.4$. The ``no evolution'' assumption would be a perfectly flat distribution, as shown by the solid red line. We also fit a linear approximation to the binned data of Figure~\ref{fig:dndzbetween}(b) of the form $d\mathcal{N}\!/dX = -0.28\,(\pm0.04) z + 0.92\,(\pm0.04)$, including the lowest redshift data point taken from~\cite{Narayanan2005}. This function shows $d\mathcal{N}\!/dX$ going to zero at $z = 3.3$, implying that no ${\MgII}$ absorbers with equivalent widths between $0.02 < W_{r}^{\lambda2796} \le 0.3$~{\AA} should be detected above this redshift. In other words, weak ${\MgII}$ absorbers are predicted not to exist at redshifts above $z = 3.3$ by this trend. This evolution follows similar trends to the evolution in cosmic metallicity and ionizing background intensity, such that we detect more weak ${\MgII}$ absorbers at low redshift, where the metallicity of the CGM is higher and the ionizing background is less likely to destroy ${\MgII}$, as opposed to higher redshifts, where the opposite applies.

\cite{Steidel1992}, and later \cite{Nestor2005}, examined the redshift evolution of $d\mathcal{N}\!/dz$ for strong ${\MgII}$ absorbers with $W_r^{\lambda2796} > 0.3$~{\AA}. They found that the number of strong ${\MgII}$ absorbers per redshift path length increases as a function of redshift from $z = 0$ to $z = 2.2$; however, they could not derive the slope of this trend to sufficient accuracy to distinguish between an evolving population or a non-evolving population. We perform a similar analysis on our sample, calculating instead $d\mathcal{N}\!/dX$, where a flat distribution implies no evolution. When we take absorbers with $W_r^{\lambda2796} > 0.3$~{\AA}, we find that a fit to the function $d\mathcal{N}\,/dX = \frac{c}{H_o}n_0\,\sigma_0(1+z)^{\epsilon}$ with a slope of $\epsilon = -0.20 \pm 0.22$ is appropriate, implying that the comoving number density and/or cross-section of strong ${\MgII}$ absorbers does not significantly evolve. However, when considering even stronger absorbers with $W_r^{\lambda2796} > 1$~{\AA}, we do observe evolution in $d\mathcal{N}\!/dX$, with the evolution parameter, $\epsilon$, becomming positive and increasing, as shown in Figure~\ref{fig:nsigmaepsilon}.

This evolution does not seem to be influenced by either an increase in the cosmic metallicity at low redshifts, or a decrease in the intensity of the cosmic ionizing background near the present epoch, which should produce more favorable conditions for ${\MgII}$ absorbing gas. In fact, in order to seemingly oppose these factors, we must infer an increase in the total quantity of ${\MgII}$ absorbing gas, observed as high equivalent width ${\MgII}$ systems, outside galaxies near $z = 2$. In Section~\ref{ch5:trendcauses}, we will explore the physical processes responsible for evolution in the universal distribution of ${\MgII}$ absorbing gas.

\subsubsection{Total Equivalent Width Distribution}
\label{ch5:totaldistro}

\begin{figure}[bth]
\epsscale{1.2}
\plotone{/Users/Nigel/PROJECTS/VULTURE/PLOTS/hist_ew_MgII_totalz.pdf}
\caption{The equivalent width distribution of all detected ${\MgII}$ absorbers, defined as the redshift path density ($d\mathcal{N}\!/dz$) in each equivalent width bin divided by the bin width. The cosmic distribution of absorbing ${\MgII}$ is well fit by a Schechter function, with the parameters detailed in Equation~\ref{eqn:schechter}.}
\label{fig:totalewdistro}
\end{figure}

\cite{Kacprzak2011MgII} combined multiple previous studies to characterize the equivalent width distribution function, $f(W_r)$. It is important to note, for comparison, that their sample with $W_r < 0.3$~{\AA} spans redshifts from $0.4 \le z \le 1.4$, while their sample with $W_r \ge 0.3$~{\AA} spans $0.4 \le z \le 2.3$. They found a Schechter function with a low equivalent width slope of $\alpha = -0.642 \pm 0.062$ and a characteristic equivalent width for the exponential cutoff of $W^* = 0.97 \pm 0.06$~{\AA} best fit the data, within the reported $1\sigma$ uncertainties. In Figure~\ref{fig:totalewdistro}, we perform the same analysis with the total sample of our survey, finding $\alpha = -0.91 \pm 0.04$ and $W^* = 1.69 \pm 0.26$~{\AA}. The point of tension between these studies lies in the value of the low equivalent width slope. \cite{Kacprzak2011MgII} combine very different surveys in order to sample the full equivalent width distribution, leading to a non-uniform data set. Specifically at the low equivalent end of the distribution, we note that the error in their determination of $\alpha$ is smaller than the scatter in the data. In private communication with the authors, they speculate that differences in the analysis methods of low equivalent width ${\MgII}$ absorption line surveys lead to unaccounted systematics and an underestimation of the error in their functional fits.

At redshifts beyond $z \sim 2.5$, the works of~\cite{Matejek2012},~\cite{Matejek2013}, and~\cite{Chen2016} described the detection statistics of strong ${\MgII}$ absorbers up to $z = 7$ using the Magellan/FIRE spectrograph. They found that $d\mathcal{N}\,/dX$ for all absorbers with $W_r^{\lambda2796} > 0.3$~{\AA} does not evolve from $z = 0.25$ to $z = 7$. However, restricting their sample to the absorbers with $W_r^{\lambda2796} > 1.0$~{\AA}, they found approximately a factor of 2 increase in number density at $z = 2-3$ compared to $z = 0$. At redshifts above $z = 3$, $d\mathcal{N}\!/dX$ declines by an order of magnitude by $z \sim 6$. Converting ${\MgII}$ equivalent widths into an effective contribution to the global star formation rate using the methods of~\cite{Menard2011} and comparing them to the total galaxy star formation rate density derived from observations of deep fields, they found exceptional agreement in the evolutionary trends of both distributions at all redshifts. The incidence of strong ${\MgII}$ absorbers at all redshifts is tied to star formation.

% Section 4.2
\subsubsection{Potential Causes for Trends}
\label{ch5:trendcauses}

The most obvious conclusion to be drawn from our analysis of ${\MgII}$ absorbers in The Vulture Survey is that the physical conditions affecting ${\MgII}$ absorbers change as a function of redshift. When we compare the sample of ${\MgII}$ absorbers at $z \sim 0.5$ to the sample of absorbers at $z \sim 2$, we find the following statements are true of the higher redshift sample:

\begin{enumerate}
\item There are more strong ${\MgII}$ absorbers per redshift path length and per absorption path.
\item The faint end slope of the equivalent width and column density distributions is flatter.
\item The ``knee'' of the Shechter fit to the equivalent width and column density distributions lies at higher values of $W_r^{\lambda2796}$ and $N(\MgII)$.
\item The cosmic mass density of ${\MgII}$ is larger.
\end{enumerate}

\noindent We can now state that the physical properties driving the global distribution of ${\MgII}$ absorbers at $z \sim 0.5$ are different than at $z \sim 2$. In addition, these physical properties do not affect all ${\MgII}$ systems equally. If we examine weak systems alone, we find that these low equivalent width absorbers follow a no-evolution expectation from present day up to $z \sim 1$, at which point they decrease in number per absorption path length rapidly until $z \sim 2.5$. If we examine the strongest systems, those with $W_r^{\lambda2796} > 1.0$~{\AA}, we find the opposite behavior, where these absorbers are more numerous at $z \sim 2.5$ than at $z \sim 0.5$. The properties of a given ${\MgII}$ absorbing cloud are governed by the nature of the radiation incident on the cloud, the gas phase metallicity, and the total amount of gas present. Therefore, possible explanations relate to the ionization conditions in the halos of galaxies at this time, the metallicity of gas located in the circumgalactic medium, and/or the quantity of metals around galaxies, which we will now explore.

% at 7ev, comoving emissivity = 300 at z=1, and 900 at z=3 - a factor of 3
% at 15ev, comoving emissivity = 0.1 at z=1, and 8 at z=3 - two orders of magnitude

\subsubsection{Ionization Conditions}
\label{ch5:ionization}

% H & M discussion
\cite{Haardt2012} represents the most recent estimate of the cosmic ionizing background as a function of redshift, which is the primary ionizing component responsible for the universal ionization state of gas in galactic halos. ${\MgII}$ absorbers are subject primarily to this UV background, with very little contribution from stellar radiation from a nearby galaxy~\citep{Churchill1999,Charlton2000,Rigby2002}. The ionization state of magnesium is governed predominantly by the ionization balance of photons of energies near 7.6~eV (0.56~Ryd), which ionizes ${\MgI}$, and near 15.035~eV (1.1~Ryd), which ionizes ${\MgII}$. The authors found that the comoving emissivity near 7.6~eV is roughly three times higher at $z = 3.1$ than at $z = 1.0$, implying that ${\MgII}$ is created more readily near $z = 3.1$. However, the comoving emissivity at 15~eV is two orders of magnitude higher at $z = 3.1$ than at $z = 1.0$, which leads to the preferential ionization of ${\MgII}$ near $z = 3.1$. The number of these ionizing 15.035~eV photons drives the ionization state of ${\MgII}$ as a function of redshift.

As the number of ionizing photons in the IGM increases with increasing redshift, holding constant the density and quantity of metals in galactic halos, we would nominally expect for the ionization parameter of ${\MgII}$ absorbers in the halos of galaxies to increase with redshift. Increasing the ionization parameter alone should decrease the observed quantity of ${\MgII}$, as it favors lower ionization parameter conditions. The comoving line density ($d\mathcal{N}\!/dX$) of ${\MgII}$ absorbers with equivalent widths below $W_r^{\lambda2796} < 0.3$~{\AA} declines linearly with increasing redshift, following the expectation that a more intense ionizing background at higher redshift disfavors ${\MgII}$ absorption. However, when examining absorbers with $W_r^{\lambda2796} > 0.3$~{\AA}, we observe the opposite trend. These strong systems are not less numerous at higher redshift as a result of harsh ionizing conditions in the halos of galaxies. We therefore disfavor the hypothesis that changes in ionzation conditions in the halos of galaxies predominantly drive the observed enhancement in the number of strong ${\MgII}$ absorbers at redshift $z \sim 2$ compared to $z \sim 0.5$.

%1~Ryd is an order of magnitude higher at $z = 1$ compared to the present epoch, and nearly two orders of magnitude larger at $z = 2$ than at present.

\subsubsection{Metallicity}
\label{ch5:metallicity}

% Metallicity discussion
The metallicity of the CGM has been best characterized by studies of damped Ly$\alpha$ absorbers (DLAs; neutral hydrogen absorbers with $\log(N(\mathrm{HI})) > 20.3$) and sub-DLAs ($19.0~\mathrm{cm^{-2}} < \log(N(\mathrm{HI})) < 20.3~\mathrm{cm^{-2}}$) located in the halos of galaxies, where many metal line transitions are observable, such as~\cite{Rafelski2012},~\cite{Quiret2016}, and~\cite{Jorgenson2013}. They found a rough trend of decreasing metallicity with increasing redshift dominated by scatter. These high column density systems certainly do not trace all ionized gas in the halos of galaxies as DLAs are predominantly neutral. Using a sample of gravitationally lensed galaxies, it has also been shown that the overall gas phase metallicity of star forming galaxies is 0.35 dex lower at $z \sim 2$ than at $z \sim 0$~\citep{Yuan2013}. In addition,~\cite{Wotta2016} and~\cite{Lehner2016} examined Lyman Limit Systems (LLS) and found that at $0.1 \le z \le 1.1$, the metallicity distribution is bimodal with peaks at $[X/H] \simeq -1.8$ and $-0.3$. At $z \ge 2$, they found a unimodal distribution with a peak at $[X/H] \simeq -2$.

Many links exist between neutral hydrogen absorption and ${\MgII}$ absorbers. Specifically,~\cite{Menard2009} show that a correlation exists between the neutral hydrogen column density of an absorber and the rest frame ${\MgII}$ equivalent width, with stronger ${\MgII}$ absorbing systems having large ${\HI}$ column densities. As ${\MgII}$ absorbers are often found associated with nearby galaxies, one can expect that the metallicity evolution of DLAs and sub-DLAs should be reflected in the evolution of ${\MgII}$ absorbing systems~\citep{Kulkarni2002,Prochaska2003,Kulkarni2005,Kulkarni2007}.

Assuming, then, that over time the overall metallicity of the circumgalactic medium increases, it would not be expected to observe larger quantities of ${\MgII}$ at $z \sim 2$ compared to $z \sim 0$. In fact, metals should build up over time in the halos of galaxies, producing more ${\MgII}$ absorption at lower redshift. We can better understand the role of metallicity by again comparing the sample of weak ${\MgII}$ absorbers to strong ${\MgII}$ absorbers. The weak systems follow the trend of rising CGM metallicity with time, by having a higher $d\mathcal{N}\!/dX$ at $z \sim 0$ as compared to $z \sim 2$. However, the strongest ${\MgII}$ systems evolve in an opposite manner to trends in cosmic metallicity, implying that other physical properties drive their evolution. Furthermore, these very high equivalent width systems have the highest column densities, and with these strong systems being more numerous at higher redshifts, they drive the cosmic matter density of ${\MgII}$ absorbing gas to its highest value at $z \sim 2$. Cosmic metallicity evolution alone, then, cannot explain the slightly enhanced cosmic mass density of ${\MgII}$ absorbing gas at $z \sim 2$ as compared to $z \sim 0$.

\subsubsection{Quantity of Magnesium}
\label{ch5:quantities}

% More stuff in halos
This leaves us, then, with the most likely conclusion being that galaxies eject more gas into their halos at $z \sim 2$ than at any other time in the form of strong, high equivalent width absorbers. \cite{Behroozi2013sfr} combined galaxy star formation rate measurements from 19 independent studies from $2006-2012$, spanning redshifts from $0 < z < 8$, to examine evolution in the cosmic star formation rate. They found that galaxies at $z \sim 2$ were forming stars at rates ten times greater than at $z = 0$. In conjunction with the fact that galactic-scale winds can be driven by star formation~\citep{Zhu2015}, it follows that galaxies were ejecting more metal-enriched gas at $z \sim 2$ than at present through supernovae-driven outflows~\citep{Sharma2013,Kacprzak2012-PA,Nestor2011}. Under the assumption that these outflows are observable as ${\MgII}$ absorption systems, they should manifenst themselves in the form of high equivalent width, high column density ${\MgII}$ absorbers, dense enough to be shielded from the more intense ionizing background at $z \sim 2$, enhanced in metallicity due to their supernovae origins. Therefore, despite competing factors such as a more intense ionizing background and lower overall cosmic metallicity at higher redshifts, we should measure a larger value of $\Omega_{\hbox{\scriptsize {\MgII}}}$ at $z \sim 2$ than at present as a result of galaxy feedback processes ejecting large quantities of ${\MgII}$ absorbing gas into their halos.

% What are weak MgII absorbers, then, if strongs are winds?
Strong ${\MgII}$ absorbers still exist at low redshift---their comoving line density does not go to zero at any point in time. However, at low redshifts near the present epoch, the majority of ${\MgII}$ absorbers are weak, low equivalent width, low column density systems whose properties are likely governed more by metallicity and incident radiation, due to their lower densities disfavoring any self-shielding. These weak absorbers may represent a population of either fragmented outflows, strong systems weathered away in hot halos~\citep{Sarkar2015}, or newly condensed gas in the halos of galaxies~\citep{Maller2004}.

Their eventual fate may be to accrete onto their host galaxy, but this rate of accretion must be small compared to the rate at which the cosmic ionizing background weakens and the cosmic metallicity of the halo increases towards lower redshifts. In other words, the rate of creation of weak ${\MgII}$ absorbers over time from $z \sim 2$ to the present epoch is greater than the rate of destruction, and this rate of creation may not be tied to an increased quantity of ${\MgII}$ absorbing gas in the halos of galaxies, but instead an evolution in cosmic properties governing the ionization state of all gas in galactic halos. We must now ask why the weakest systems build up slowly from $z > 2$ to the present epoch, directly opposing the trends observed for the strongest systems. To answer this question, we must understand the innate physical similarities and differences between high equivalent, high column density absorbers and low equivalent width, low column density absorbers.

% Stuff's within the Virial Radius
Studies associated with COS-Halos~\citep{Tumlinson2011} seeking to understand the distribution of metals around galaxies have found the majority of cool, metal absorbing gas lies within the virial radius of galaxies~\citep{Peeples2014}. \cite{Stern2016} also found that the mean cool gas density profile around galaxies scales as $R^{-1}$, with most strong, low ionization metal absorbers existing near the galaxy itself. Further refining these assertions specifically for ${\MgII}$,~\cite{Churchill2013letter} found that the covering fraction of ${\MgII}$ absorbers falls to $0$ at projected distances beyond the virial radius. This means that the ${\MgII}$ absorbers observed in The Vulture Survey likely lie within the virial radius of a galaxy. The ${\MgII}$ absorbing gas observed near $z \sim 2$ should eventually fall onto its host galaxy, or stall at large radii and remain a permanent part of the host galaxy halo~\citep{Oppenheimer2010,Ford2014}. What we observe in the form of low redshift ${\MgII}$ absorbers therefore may not only be the result of outflows near the present epoch, but also the remnants of the most energetic star-forming period in cosmic history.

% Strong absorbers are wind-related.
Perhaps the most clear-cut evidence that many very strong ($W_r^{\lambda2796} > 1.0$~{\AA}) absorbers have their origins in star formation driven winds comes from the works of~\cite{Matejek2012} and~\cite{Chen2016}. The authors examined their sample of high equivalent width ${\MgII}$ absorbers spanning redshifts $2 < z < 7$, combined with samples from~\cite{Nestor2005} and~\cite{Prochter2006} with data below $z < 2$, to relate the ${\MgII}$ population distribution with the cosmic star formation rate density. Following the methods of~\cite{Menard2011}, they convert the ${\MgII}$ rest frame equivalent widths to a comoving ${\OII}$ luminosity density, and from there derive a star formation rate density. When these ${\MgII}$-derived star formation rate densities were compared with the mean galaxy specific star formation rates from $0 < z < 7$, they found remarkable agreement. We therefore corroborate their results, finding good agreement between the evolutionary behavior of $d\mathcal{N}\,/dX$ for strong ${\MgII}$ absorbers and the cosmic star formation rate density. These high equivalent width ${\MgII}$ systems are preferentially produced by feedback processes associated with star formation.

% Number of weak MgII absorbers at low z is increasing, not the cross-section
Referring back to Figure~\ref{fig:nsigmaepsilon}(a), we can examine how $\frac{c}{H_o}n_0\,\sigma_0$ changes for absorbers of different equivalent widths. Here, we can infer that the physical parameter causing $d\mathcal{N}\!/dX$ to fall between $z \sim 0$ to $z \sim 2$ is the number density of weak ${\MgII}$ absorbers in galaxy halos ($n_0$), instead of the absorbing cross-section ($\sigma_0$). In~\cite{Evans2013}, the authors use Cloudy~\citep{Ferland2013} to estimate the size and number densities of clouds giving rise to ${\MgII}$ absorption. According to these simulations, absorber sizes are roughly an order of magnitude smaller at $z \sim 0$ than they were at $z \sim 2$ for clouds of the same hydrogen density across a wide range of densities. This implies the absorbing cross-section is lower at present than at $z \sim 2$, and, therefore, the number density of weak ${\MgII}$ absorbers should be the driving factor leading to the enhancement of $d\mathcal{N}\!/dX$ for low equivalent width systems at the present epoch.

% Cooling radius discussion
Furthermore,~\cite{MAGIICAT3} discussed how the theoretical cooling radius, the radial distance from the center of a halo at which the initial gas density equals the characteristic density at which gas can cool, may serve as an important factor in determining the observational properties of ${\MgII}$ absorbers. They found that within the cooling radius, the covering fraction of ${\MgII}$ absorption increases, and does so most dramatically for absorbers with equivalent widths below $W_r^{\lambda2796} = 0.3$~{\AA}. At equivalent widths above this level, covering fractions inside and outside the cooling radius are statistically similar. This points toward a possible physical origin for a substantial portion of weak, low equivalent width ${\MgII}$ absorbers---these clouds cool and condense from collapsing gas clouds within the cooling radius. This gas could have originated from fragmented outflows or infalling material which, over time, met the conditions required to cool enough to be observed as ${\MgII}$, instead of a higher, more ionized species.

% The big picture
Given the above considerations, we favor a picture where galaxies, during the most active epoch of star formation in cosmic history, expell large quantities of metal enriched gas into their halos through star formation driven outflows at $z \sim 2$. These outflows manifest themselves in observations as strong ${\MgII}$ absorbers. Their destiny will be to eventually either fall back onto the galaxy and enrich the ISM or remain in the halo, subject to the cosmic ionizing background and possible fragmentation. Those absorbers which remain intact and accrete onto the galaxy are lost from observations at lower redshifts, while those which remain at larger radii will likely reach equilibrium with the ionization conditions in the halo. As the intensity of the cosmic ionizing background decreases with time, and more metals are ejected into the halo over time, more of this gas appears as weak ${\MgII}$ absorption near $z \sim 0$. Many of these weak ${\MgII}$ absorbers are likely more passive in nature, participating less in the overall baryon cycle in and out of galaxies, subject primarily to the environmental conditions of the CGM.
  % Chapter 5

\clearpage
\resetcounters

\section{\MakeUppercase{Properties and Evolution of {\CIV} Absorbers}}
\label{ch6}

Now we can finally get down to the science!

\subsection{Number of Absorbers Per Path Length}
\label{ch6:dndx}

Dat path doe.

\subsection{Parameterizing $dN\,/dX$ to Derive Physical Properties}
\label{ch6:parameterize}

Describe how we interpret evolution in terms of $n$, $\sigma$, and $\epsilon$.

\subsection{Equivalent Width Distribution}
\label{ch6:ewdistro}

Schechter that shit.

\subsection{Column Density Distribution}
\label{ch6:columndistro}

Schechter it HARDER.

\subsection{$\Omega_{\CIV}$}
\label{ch6:omega}

Such cosmology.

\subsection{Strong vs. Weak Absorbers}
\label{ch6:strongweak}

Here's where the cool stuff really lies.
  % Chapter 6

\clearpage
\resetcounters

\section{\MakeUppercase{Kinematics Properties using TPCF Analysis}}
\label{ch7}

Here we do the whole Nikki thing and interpret the results.

\subsection{{\MgII} Kinematics}
\label{ch7:MgII}

Here we slice it, we dice it, and we look at a lot of TPCFs.

\subsubsection{{\MgII} Redshift Evolution}
\label{ch7:MgIIredshift}

Slice with respect to z.

\subsection{{\MgII} Optical Depth Behavior}
\label{ch7:MgIIoptical}

Slice with respect to tau.

\subsection{{\CIV} Kinematics}
\label{ch7:CIV}

Here we slice it, we dice it, and we look at a lot of TPCFs.

\subsubsection{{\CIV} Redshift Evolution}
\label{ch7:CIV}

Slice with respect to z.

\subsection{{\CIV} Optical Depth Behavior}
\label{ch7:CIV}

Slice with respect to tau.
  % Chapter 7

\clearpage
\resetcounters

\section{\MakeUppercase{Conclusions}}
\label{ch8}

WHAT DOES IT ALL MEAN?!

\subsection{{\MgII}}
\label{ch8:MgII}

Long song and dance about how MgII is so great and all.

\subsubsection{Strong Absorbers}
\label{ch8:MgIIstrong}

Talk about all the properties of strong absorbers.

\subsubsection{Weak Absorbers}
\label{ch8:MgIIweak}

Talk about all the properties of weak absorbers.

\subsubsection{Kinematics}
\label{ch8:MgIIkinematics}

Talk about kinematic properties of MgII absorbers.

\subsection{{\CIV}}
\label{ch8:CIV}

Even longer song and dance about how CIV is arguably even more interesting and important.

\subsubsection{Strong Absorbers}
\label{ch8:CIVstrong}

Talk about all the properties of strong absorbers.

\subsubsection{Weak Absorbers}
\label{ch8:CIVweak}

Talk about all the properties of weak absorbers.

\subsubsection{Kinematics}
\label{ch8:CIVkinematics}

Talk about kinematics properties of CIV absorbers.

\subsection{Evolution in the Context of Galaxy Evolution}
\label{ch8:evolution}

Evolution.

\subsection{Consequences, and Verification}
\label{ch8:consequences}

Consequences.

\subsection{Future Work}
\label{ch8:futures}

This is where my postdoc stuff would go...IF I HAD ONE.
  % Chapter 8

%% Repeat as many chapters as you need with
% \clearpage
% \resetcounters
% \input{chapterX}

\clearpage

% Below is for an appendix or appendices if you have them. If not just comment it all out

\pagestyle{empty}  % No page number on the half-title appendix page...okie dokie can do nmsu. (Note the page number count still increases, we're just printing the page number on the page.)

\begin{center}
APPENDIX  % NMSU requires a half-title page that is labeled APPENDIX or APPENDICES (depending if you have 1 or 2+ appendix/appendices.) Yes, literally a whole page with nothing but this word, cause ya know, that makes sense...
\end{center}

\addtocontents{toc}{\protect\contentsline{}{}{}}  % Add an extra line before 'Appendices' in the table of contents
\appendix  % Start the appendix

\clearpage
\resetcounters

\pagestyle{plain}  % Resume printing page numbers

\label{appendix}
\section{\MakeUppercase{There Be Math in Them There Hills}}

In this appendix we further explain the line detection technique introduced in Section~\ref{ch3:detection}. Equations show we can match the filters.

\begin{equation}
P(F,T) = \frac{F}{F_o} - \int_{T_D}^{0}{\frac{T}{T_D}dT}
\end{equation}

where $P$ is the probability of me giving a Fuck at a given point in time, $F$ is the number of Fucks I currently have, $F_o$ is the total number of Fucks available, $T$ is the current time remaining until a deadline, for a given total given to complete the task, $T_D$.

\subsection{Appendix subsection}

Did you know that this won't show up as listed in the table of contents, in accordance with NMSU thesis policy? Exciting.
  % Appendix A - appendixA.tex

%\clearpage
%\resetcounters

%\input{appendixB}  % Appendix B - appendixB.tex


%% Repeat as many appendices as you need with
% \clearpage
% \resetcounters
% \input{appendixX}

\clearpage

\singlespace
\addcontentsline{toc}{section}{REFERENCES}  % This line makes sure that REFERENCES shows up in the table of contents with the correct page number
\bibliographystyle{apj}  % You can use any bibliography style you like, but hey, I like ApJ cause I'm used to it
\bibliography{bibliography.bib}  % Your bibtex file with all that lovely natbib info

\end{document}
