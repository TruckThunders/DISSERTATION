\section{\MakeUppercase{Introduction}}
\label{ch1}

The field of extragalactic astronomy has blossommed in the past decade, with the advent of large telescopes, grand uniform surveys, and the computing power to simulate complex processes on universal scales. By combining observational constraints and cosmological hydrodynamic simulations, we have managed to create galaxies that look like those we have observed for decades. However, as we make more detailed observations revealing the specifics of ever-increasingly complex physical processes, we find that the models do not truly mimic reality. In the end, progress can be made through more detailed observations and more complex simulations.

We know so much more about the material inside galaxies than we do about the material outside galaxies. There exist large reservoirs of gaseous material outside the visible extent of galaxies, in their halos, which must be studied in detail in order to understand how galaxies grow and evolve. In addition, as our understanding of all forms of matter evolves, so must our analysis methods. Individual case studies still have their merit; however, due to the sheer volume of our observable universe and the wild variance in all properties of observable structures, detailed surveys are the next step forward in understanding exactly how our universe works.

This work attempts to begin that process by surveying in great detail the wealth of archived observations at the world's largest, most productive observatories. These data sets make no concessions in resolution or detection sensitivity, and hope to characterize the properties of matter in galaxy halos in great detail so as to trace and predict the full spectrum of gaseous structures throught the universe.

\subsection{Galaxy Formation and Evolution}
\label{ch1:galaxyevolution}

One of the most elusive yet important goals in extragalactic research aims to discover how galaxies form, grow, and evolve in an effort to truly understand our place in the universe. The creation of a galaxy begins with a cosmic overdensity, where matter congregates to begin the process of building complex structures of stars, gas, and dust. Feedback from star formation and AGN, along with mergers, shape the galaxy and its surroundings into complex environments. The stage of most uncertainty throughout this process involves feedback, the injection of energy into the gaseous medium within and outside of galaxies which, as has been shown in simulations, is required to produce realistic looking galaxies.

The Milky Way serves as the closest and best example of a galaxy. However, galaxies come in all kinds of different shapes and sizes. To truly understand how we came to be, as humans on the planet Earth, we must work to understand how our galaxy came to be, and what processes played part in creating the local environment we live in today.

\subsection{The Circumgalactic Medium}
\label{ch1:cgm}

Characterizing the baryonic gas processes within and surrounding galaxies is central to understanding their formation and evolution. Quantifying the spatial extent, kinematics, and, in particular, the recycling and/or escape fraction of circumgalactic gas are of primary importance in that they place direct observational constraints on simulations of galaxies and provide insights into the workings of galaxy evolution.

In general, the gas structures in and around galaxies can be divided into three broad categories: the interstellar medium (ISM), the circumgalactic medium (CGM), and the intergalactic medium (IGM). The CGM, being the gas reservoir that interfaces with the star-forming ISM, outflowing stellar-driven winds, and the accreting IGM, may contain up to 50\% of the baryonic mass bound to galaxies \citep{Tumlinson2011} and account for up to 50\% of the baryons unaccounted for in galaxy dark matter halos \citep{Werk2014}.  As such, the CGM may play the most critical role in governing the properties of galaxies \citep[e.g.,][]{Oppenheimer2010, MAGIICAT3}, including regulatory physics leading to the observed stellar mass function \citep[e.g.,][]{Behroozi2013} and the stellar mass-ISM metallicity relationship \citep[e.g.,][]{Tremonti2004}.

The physical extent of the CGM and the transition zone between the CGM and IGM are currently open questions. Studies by \citet{Steidel2010}, \citet{Prochaska2011}, and \citet{Rudie2013} indicate a transition from the CGM to the IGM at $\log N({\HI}) \simeq 14$ and a projected distance of $\sim\! 300$~kpc from galaxies at $z\sim 2.5$.  At this redshift, $\log N({\HI}) \simeq 14$ corresponds to an overdensity of $\log \rho_{\hbox{\tiny H}}/\bar{\rho}_{\hbox{\tiny H}} \simeq 0.5$, whereas at $z\simeq 0$, this overdensity would suggest a CGM/IGM transition at $\log N({\HI}) \simeq 13$ \citep[see][]{Dave1999}.  Indeed, \citet{Ford2013mass} show that, in the over-dense regions hosting galaxies, the extent of the {\HI} at fixed column density increases with virial mass, suggesting that a single fiducial physical size for the CGM may not apply across the entire mass range of galaxies; it may be more appropriate to scale CGM properties relative to the virial radius \citep[e.g.,][]{CWC2013Masses,MAGIICAT3}.

%The cgm is pretty big. Like, bigger than most things you know about.

\subsection{The Baryon Cycle}
\label{ch1:baryoncycle}

One of the most important questions in modern studies of galactic evolution asks, how do baryons cycle into and out of galaxies, and how does this cycle determine the growth and evolution of galaxies themselves? More specifically, how does the process of gas accretion, star formation, and subsequent supernovae-driven feedback shape both the galaxies themselves and their circumgalactic medium (CGM)? By using spectroscopic observations of quasars, we can identify and analyze metal line absorbers in and around the halos of foreground galaxies. Though absorption line studies by themselves cannot directly answer these questions, the statistical results from such studies can provide vital information from which further progress can be made.

%Baryons go in, more baryons come out. YOU can't explain that.

\subsection{Quasar Absorption Line Spectroscopy}
\label{ch1:absorptionlines}

The details of this complex process should be observable in the relationship between the stars in the galaxy and the gas phase material located between the stars within the galaxy (interstellar medium; ISM), just outside the galaxy (circumgalactic medium; CGM), and between galaxies themselves (intergalactic medium; IGM). Unfortunately, detailed observations of this gas are difficult as it remains mostly neutral (CITATION), and therefore does not strongly emit. Therefore, massive efforts have been undertaken to observe this gas in absorption using the most advanced telescopes.

A major problem in studying galaxy evolution at high redshifts is that faint, low mass galaxies cannot be observed with current facilities. These smaller, more numerous galaxies pose major problems when attempting to characterize the ionization conditions of the early universe and detailing the growth history of galaxies. They are all but invisible due to their low luminosities, which makes it nearly impossible to completely inventory galaxy populations and find modern day galaxy analogs at high redshift. Studying properties of galaxy evolution, then, must involve a luminosity-independent tracer of the processes which grow and affect galaxies across cosmic time. One such technique involves observing absorbing gas in quasar spectra, effectively measuring properties of the shadows of gaseous structures around galaxies.

Quasars are exceptionally bright objects, often found at large distances, or high redshift. Their luminosities can exceed SOMETHING BIG, and they can be observed at distances corresponding to light travel times near to that of the beginning of the universe. They serve as extreme cosmic lighthouses, illuminating material located between observers and the quasars themselves.

By taking a spectrum of a quasar, we can learn about material at large distances which may not emit light of its own. By absorbing the light of the quasar at specific wavelengths, this mater, or gas, imprints onto the spectrum a characteristic absorption feature. These absorption features can be measured in order to determine their redshift, and also their underlying physical properties.

\subsection{A Brief History of {\MgII} Absorption Surveys}
\label{chp1:mgiibackground}

One of the most prolific absorption features, the {\MgIIdblt} doublet, traces cool ($T \simeq 10^4~\mathrm{K}$; \cite{Churchill2003}) metal enriched gas in the disks and halos of galaxies. It is one of the best tracers of this gas because it can exist in a wide range of ionizing conditions, ranging in ionization parameter from $-5 < \log U < 1$~\citep{Churchill1999}, it is observable in optical wavelengths for redshifts between $0.1 < z < 2.6$, and it has predictable line characteristics defined by its resonant doublet nature which make it ideal for automated searches.

The origin of ${\MgII}$ absorbing gas is still debated. As summarized in~\cite{Kacprzak2011} and~\cite{Matejek2013}, two separate interpretations exist to explain the origin of strong, high equivalent width ($W_r$) {\MgII} absorbers ($W_r^{\lambda2796} > 0.3$~{\AA}) and weak, low equivalent width ($W_r^{\lambda2796} < 0.3$~{\AA}) ${\MgII}$ absorbers.

For the strong, higher equivalent width systems, multiple correlations exist between the rest frame ${\MgII}$ equivalent width around galaxies and the host galaxy's star formation properties. \cite{Zibetti2007}, \cite{Lundgren2009}, \cite{Noterdaeme2010}, \cite{Bordoloi2011}, and \cite{Nestor2011} all found a correlation betwen $W_r^{\lambda2796}$ and blue host galaxy color, showing that galaxies with more active star formation have more metal enriched gas in their halos. \cite{Bordoloi2014} also found that ${\MgII}$ equivalent width increases with increasing star formation rate density. In addition, spectroscopic observations of star forming galaxies have revealed strong ${\MgII}$ absorption blueshifted $300 - 1000$~{\kms} relative to the host galaxy~\citep{Tremonti2007,Weiner2009,Martin2009,Rubin2010}.

Multple correlations have also been found between the equivalent width of strong ${\MgII}$ absorbers and host galaxy mass. \cite{Bouche2006} found an anti-correlation between galaxy halo mass, derived from the cross-correlation between ${\MgII}$ absorption systems and luminous red galaxies, and ${\MgII}$ equivalent width, showing that individual clouds of a ${\MgII}$ system are not virialized in the halos of galaxies. They interpreted their results as a strong indication that high equivalent width absorbers with $W_r^{\lambda2796} \gtrsim 2$~{\AA} arise in galactic outflows. Marginal anti-correlations between ${\MgII}$ equivalent width and galaxy halo mass using the same cross-correlation method were also reported by~\cite{Gauthier2009} and~\cite{Lundgren2009}. It is imporant to note, however, that~\cite{Churchill2013letter} and~\cite{MAGIICAT3} find no correlation between $W_r^{\lambda2796}$ and halo mass when halo mass is derived from abundance matching. They instead find that galaxies inhabiting more massive dark matter halos have stronger absorption at a given distance.

For the weak, lower equivalent width systems, it seems none of the above correlations hold. \cite{Chen2010b}, \cite{Kacprzak2011}, and \cite{Lovegrove2011} found little evidence for a correlation between galaxy color and ${\MgII}$ equivalent width when restricting their samples to weak absorbers. \cite{Kacprzak2011} measured the orientation of galaxies relative to ${\MgII}$ detections in the sight lines of background quasars and identified low metallicity, low equivalent width ${\MgII}$ absorbers co-planar with some some galaxy disks, implying structures associated with accreting filaments as opposed to outflows, which are more often observed perpendicular to the galaxy disk~\citep{Bordoloi2011,Kacprzak2012-PA,Bouche2012}. Finally, the simulations of~\cite{Stewart2011} and~\cite{Ford2013mass} revealed a reservoir of low-ionization, metal enriched, co-rotating gas around massive galaxies. Together, these studies imply that weak ${\MgII}$ absorption systems may preferentially trace low metallicity infall and co-rotating gas in the circumgalactic medium.

\cite{MAGIICAT1} constructed a sample of ${\MgII}$ absorbers and their associated galaxies and examined both strong and weak ${\MgII}$ absorbers from $0.07 \le z \le 1.1$. In the subsequent analysis of their sample,~\cite{MAGIICAT2} found a more extended ${\MgII}$ absorbing CGM around higher luminosity, bluer, higher redshift galaxies. In addition, in~\cite{MAGIICAT4}, they found that bluer galaxies replenish their ${\MgII}$ absorbing CGM through outflows, whereas red galaxies do not. Finally, in~\cite{MAGIICAT5}, it is made clear that the largest velocity dispersions in ${\MgII}$ absorbing systems are measured around blue, face-on galaxies probed along their minor axis, strongly suggesting that these ${\MgII}$ absorbers originate in bi-conical outflows.

Many surveys have been undertaken to inventory ${\MgII}$ absorbers and examine their evolution. The earliest studies~\citep{Lanzetta1987,Tytler1987,Sargent1988,Steidel1992} found that ${\MgII}$ systems with rest equivalent widths above $0.3$~{\AA} show no evolution in $d\mathcal{N}\!/dz$ between redshifts $0.2 < z < 2.15$. These studies also found that the equivalent width distribution function, $f(W_r^{\lambda2796})$, could be fit equally well with either an exponential or a power-law function. It remains uncertain whether the cosmic distribution of ${\MgII}$ in galactic halos exhibits a fractal, self-similar nature, or if $f(W_r^{\lambda2796})$ flattens at equivalent widths below $W_r^{\lambda2796} < 0.3$~{\AA}.

${\MgII}$ absorption surveys have taken one of two different approaches to try to analyze the global distribution of ${\MgII}$ absorbing gas across cosmic time. \cite{Churchill1999} and \cite{Narayanan2007} aimed to determine more precisely how $d\mathcal{N}\!/dz$ and $f(W_r^{\lambda2796})$ evolve with redshift by surveying weak ${\MgII}$ absorbers. They found that, for these low equivalent width absorbers, $d\mathcal{N}\!/dz$ increases as a function of increasing redshift up until $z = 1.4$. At higher redshifts, $d\mathcal{N}\!/dz$ falls to lower values, though the uncertainties are large. In addition, they found the equivalent width distribution function for weak absorbers is best fit by a power-law, strongly disfavoring an exponential fit to the overall distribution.

The most recent studies have employed new multi-object spectrographs such as the Sloan Digital Sky Survey (SDSS) and the FIRE spectrograph on the Magellan Baade Telescope~\citep{Nestor2005,Matejek2012,Chen2016}. \cite{Nestor2005}, who examined over 1300 intervening ${\MgII}$ absorbers in SDSS quasar spectra with $W_r^{\lambda2796} > 0.3$~{\AA}, found that the equivalent width distribution function is well fit by an exponential. They did not find evidence for redshift evolution in systems with $0.4 < W_r^{\lambda2796} < 2$~{\AA}, but observed an enhancement in the number of $W_r^{\lambda2796} > 2$~{\AA} absorbers per comoving redshift path length as a function of increasing redshift from $z \sim 0$ up to $z \sim 2$. \cite{Matejek2012} and~\cite{Chen2016}, analyzing 279 ${\MgII}$ absorbing systems from $2 < z < 7$ in infrared FIRE spectra, also found that the equivalent width distribution function is well fit by an exponential. They also observed that systems with $W_r^{\lambda2796} < 1.0$~{\AA} show no evolution with redshift, but higher equivalent width systems grow in number density from low redshift to $z \sim 3$, after which the number density declines. Collectively, these surveys imply physical changes in the astrophysical processes or in the state of the gas structures in the environments giving rise to ${\MgII}$ absorption as the universe ages.

% "EW THAT'S MADE OF DEAD TREES"

\subsection{A Brief History of {\CIV} Absorption Surveys}
\label{chp1:civbackground}

The first study to characterize the distribution of intervening CIV absorbers was~/cite{Young1982}, which determined that these lines were randomly distributed in redshift in a manner consistent with absorption from intervening galaxies. /cite{Sargent1988} also showed that the number density of systems per unit redshift decreases with increasing redshift from $1.8 /le z_{em} /le 3.56$. /cite{Steidel1990}, examining strong CIV absorbers at redshifts between $1.3 /le z_{abs} /le 4.0$, found also that the number of CIV absorption systems per unit redshift range decreases with increasing redshift, in a manner inconsistent with a constant comoving density of absorbers. Therefore, they stated, the properties of CIV absorbers must evolve over time.

In terms of kinematics,~/cite{Steidel1990} noted that the peak in the CIV two-point correlation function on velocity scales $200 /le \Delta v /le 600~{\kms}$ appears to have the same power at low and high redshift. ~/cite{Rauch1996} examined the velocity structure of intervening CIV absorbers in the spectra of 3 intermediate redshift quasars and detrmined that gas temperatures were likely near $3.8 /times 10^4~/mathrm{k}$, and the TPCF of CIV systems suggests that there is more than one source of velocity dispersion. They interpret the shape of the resulting TPCF as owing to ensembles of objects with the kinematics of dwarf galaxies on a small scale, while following the Hubble flow on a larger scale. % Reword, as some of these statements are word for word out of the papers they came from.

\cite{Churchill1999} examined data from the Hubble Space Telescope Faint Object Spectrograph for {\CIV} absorbers associated with known {\MgII} absorbers measured in high resolution Keck/HIRES data. They noticed a strong correlation between {\MgII} kinematics and {\CIV} equivalent width. They interpreted that this correlation could imply a connection between outflowing {\MgII} clouds and higher ionization state halo gas.

\cite{Rauch2001} measured the kinematics and column density differences of {\CIV} absorption systems along the lines of sight to three gravitationally lensed quasars. They found the spatial distribution of the gas to be mostly featureless with detectable velocity shear up to $\sim70~\mathrm{kms}$. They found the clouds quiescent, in that the energy transmitted to the gas as measured by the amount of turbulence derived along the line of sight, and therefore determined that the absorbing structures are not internal to galaxies. They posited that {\CIV} absorbers could be gas expelled recently or pre-enriched from earlier star formation.

\cite{Chen2001} examined 14 galaxy absorber pairs, along with 36 galaxies without associated {\CIV} absorption lines. They found that {\CIV} absorption line systems cluster strongly on velocity scales of $v \lesssim 250~\mathrm{\kms}$ and impact parameter scales of $\rho \lesssim 100~h^{-1}~\mathrm{kpc}$. In addition, they note that galaxies of all morphological types and luminosities can possess extended gaseous envelopes, with covering factors near unity, at impact parameters less than $100~\mathrm{kpc}$. They concluded that accreting satellites are the most likely sources of this metal enriched halo gas.

\cite{Simcoe2004} directly measured the metallicity distribution function for the $z \sim 2.5$ intergalactic medium in the spectra of seven quasars. They found no evidence for a universal metallicity floor, as had been suggested for some scenarios of Population III enrichment in the early universe. In addition, they found no trends in metallicity as a function of IGM density.

\cite{Fox2007} studied 63 damped Lyman-$\alpha$ (DLA) systems and 11 sub-DLAs with associated {\CIV} absorption. They detected {\CIV} clouds moving at velocites in excess of the escape speed, determined by measuring the total line width of the neutral gas profile, in roughly 40\% of their systems. They inferred that these clouds may arise in high ionized outflowing winds, powered by galaxies with star formation rates (SFR) of $\sim 2~M_{\odot}~\mathrm{yr^{-1}}$.

\cite{Shull2014} used {\it HST}/COS spectra of background quasars to measure elemental abundances in the low redshift intergalactic medium. They found that {\CIV} has increased in abundance by a factor of 10 from $z \sim 5$ to present.

\cite{Bordoloi2014} measured {\CIV} absorption in the COS-Dwarfs survey, probing the gaseous halos near 43 low-mass galaxies at $z \ge 0.1$. They detected {\CIV} out to $\sim 100~\mathrm{kpc}$, roughly $0.5 R_{vir}$, from the host galaxies. They also reported a tentative correlation between {\CIV} equivalent width and SFR, and concluded that energy-driven star formation winds must expel into the CGM a comparable mass to that of the carbon found in the stars of these galaxies.

\subsection{Project Goals}
\label{chp1:goals}

The goal of this project is to wrest my PhD from the icy grips of my faculty captors.
