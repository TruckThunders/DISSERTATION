\section{\MakeUppercase{Introduction}}
\label{ch1}

Testing \ref{ch3}

\subsection{Quasar Absorption Line Spectroscopy}
\label{ch1:absorptionlines}

One of the most important questions in modern studies of galactic evolution asks, how do baryons cycle into and out of galaxies, and how does this cycle determine the growth and evolution of galaxies themselves? More specifically, how does the process of gas accretion, star formation, and subsequent supernovae-driven feedback shape both the galaxies themselves and their circumgalactic medium (CGM)?

A major problem in studying galaxy evolution at high redshifts is that faint, low mass galaxies cannot be observed with current facilities. These smaller, more numerous galaxies pose major problems when attempting to characterize the ionization conditions of the early universe and detailing the growth history of galaxies. They are all but invisible due to their low luminosities, which makes it nearly impossible to completely inventory galaxy populations and find modern day galaxy analogs at high redshift. Studying properties of galaxy evolution, then, must involve a luminosity-independent tracer of the processes which grow and affect galaxies across cosmic time. One such technique involves observing absorbing gas in quasar spectra, effectively measuring properties of the shadows of gaseous structures around galaxies.

Quasars are exceptionally bright objects, often found at large distances, or high redshift. Their luminosities can exceed SOMETHING BIG, and they can be observed at distances corresponding to light travel times near to that of the beginning of the universe. They serve as extreme cosmic lighthouses, illuminating material located between observers and the quasars themselves.

By taking a spectrum of a quasar, we can learn about material at large distances which may not emit light of its own. By absorbing the light of the quasar at specific wavelengths, this mater, or gas, imprints onto the spectrum a characteristic absorption feature. These absorption features can be measured in order to determine their redshift, and also their underlying physical properties.

In \S\ref{whalesec} we talk in extensive detail about the biology and presumed intelligence of the discovered whales.

We conclude that we should worship these exowhales as our benevolant overlords, (see Chapter~\ref{chap2}).

\subsection{Galaxy Formation and Evolution}
\label{ch1:galaxyevolution}

The Milky Way serves as the closest and best example of a galaxy. However, galaxies come in all kinds of different shapes and sizes. To truly understand how we came to be, as humans on the planet Earth, we must work to understand how our galaxy came to be, and what processes played part in creating the local environment we live in today.

It is theorized by \citet{Haardt2012} that something crazy happens. THIS IS A TEST CItation for now\footnote{Testing out footnotes here as well}.

\subsection{The Circumgalactic Medium}
\label{ch1:cgm}

The cgm is pretty big. Like, bigger than most things you know about.

\subsection{The Baryon Cycle}
\label{ch1:baryoncycle}

Baryons go in, more baryons come out. YOU can't explain that.
