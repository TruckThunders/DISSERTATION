\begin{center}
ABSTRACT
\end{center}

\bigskip

\begin{center}
The Vulture Survey: Feasting on the Bones of Archival Spectra Left to Die\\
\bigskip
BY\\
\bigskip
NIGEL L. MATHES, B.S., M.S.
\end{center}

\bigskip

\begin{center}
Doctor of Philosophy\\
New Mexico State University\\
Las Cruces, New Mexico, 2012\\
Dr. Christopher W. Churchill, Chair
\end{center}

\bigskip
%\bigskip

% ABSTRACT MAY NOT EXCEED 350 WORDS

We present detailed measurements of the redshift path density, equivalent width distribution, column density distribution, and redshift evolution of ${\MgII}$ and ${\CIV}$ absorbers as measured in archival spectra from the UVES spectrograph at the Very Large Telescope (VLT/UVES) and the HIRES spectrograph at the Keck Telescope (Keck/HIRES) to equivalent width detection limits below $0.01$~{\AA}. This survey examines 432 VLT/UVES spectra from the UVES SQUAD collaboration and 170 Keck/HIRES spectra from the KODIAQ group, representing 580 unique sightlines, allowing for detections of intervening ${\MgII}$ absorbers spanning redshifts $0.1 < z < 2.6$ and intervening ${\CIV}$ absorbers spanning redshifts $1 < z < 5$. We employ an accurate, automated approach to line detection which consistently detects redshifted absorption doublets. We find that an increased number of high column density ${\MgII}$ and ${\CIV}$ absorbers at $z = 2$ drives an overall increase in the quantity of metals around galaxies as compared to the present epoch. We conclude that galaxies eject more metal enriched gas into their halos around $z = 2$ than at any other redshift through star formation driven outflows. We determine that weak ${\MgII}$ and ${\CIV}$ absorbers, those with equivalent widths less than $0.3$~{\AA}, are physically distinct and evolve separately from very strong absorbers, which have equivalent widths greater than $1.0$~{\AA}. Over this same time period, evolving ionizing conditions in the halos of galaxies gives rise to an increasing population of low equivalent width, passive ${\MgII}$ and ${\CIV}$ absorbers. From $z = 2$ to the present, feedback processes decline and we observe fewer very strong systems. From $z = 2$ to $z = 5$, SOMETHING ELSE HAPPENS. EXPLAIN THIS SOMETHING ELSE.
