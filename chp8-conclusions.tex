\section{\MakeUppercase{Conclusions}}
\label{ch8}

%WHAT DOES IT ALL MEAN?!

Using archival data from VLT/UVES and Keck/HIRES, we have undertaken the most complete survey of ${\MgII}$ absorbers in 602 quasar spectra with high resolution ($\sim 7~\mathrm{\kms}$), allowing for the detection of both strong and weak {\MgII} absorbsers. Our survey spans absorption redshifts from $0.18 < z < 2.57$, allowing for characterization of the evolution of the distribution of these absorbers across cosmic time. Using our own detection and analysis software, we are able to accurately characterize the equivalent width detection limit, absorption path length, and survey copmleteness to a level allowing for an accurate determination of $d\mathcal{N}\!/dz$, the equivalent width distribution function, the column density distribution function, and the total cosmic mass density of {\MgII} absorbers. Our main findings are as follows:

\begin{enumerate}
\item We find 1180 intervening ${\MgII}$ absorption line systems with equivalent widths from $0.003$~{\AA} to $8.5$~{\AA}, and redshifts spanning $0.14 \le z \le 2.64$.
\item We present the distributions of the number of absorbers per unit redshift, $d\mathcal{N}\!/dz$, and the comoving ${\MgII}$ line density, $d\mathcal{N}\!/dX$, as a function of minimum equivalent width threshold and as a function of redshift. We parameterize the evolution in $d\mathcal{N}\!/dX$ specifically with an emperical fit to the distribution in the form of $d\mathcal{N}\,/dX = \frac{c}{H_o}n_0\,\sigma_0(1 + z)^{\epsilon}$, showing $\epsilon$ increases from $\epsilon=-1.11$ for all absorbers with $W_r^{\lambda2796} \ge 0.01$~{\AA} to $\epsilon=0.88$ for absorbers with $W_r^{\lambda2796} \ge 2$~{\AA}. High equivalent width ${\MgII}$ absorbers decrease in relative number per absorption path length from $z = 2$ to present, and low equivalent width ${\MgII}$ absorbers increase in relative number per absorption path length from $z = 2$ to present. We observe no evolution with redshift in absorbers with equivalent widths between $0.2 < W_r^{\lambda2796} < 1$~{\AA}.
\item We derive a closed form analytic parameterization of $d\mathcal{N}\!/dX$ for all ${\MgII}$ absorbers. As shown in Equation~\ref{eqn:dndxanalytic}, $d\mathcal{N}\!/dX$ can be expressed as a function of $W_{r,\mathrm{min}}^{\lambda2796}$ and $z$. In this parameterization, we also show how the comoving Hubble optical depth, $\frac{c}{H_o}n_0\,\sigma_0$, and evolution parameter, $\epsilon$ evolve with increasing minimum sample equivalent width (see Figure~\ref{fig:nsigmaepsilon}). These form a physical basis for understanding how the comoving line density of ${\MgII}$ absorbers evolves over cosmic time.
\item The equivalent width distribution function and the column density distribution function for ${\MgII}$ absorbers are both well fit by a Schecter Function, with a characteristic normalization, faint end slope, and exponential cutoff. Both functions show redshift evolution, specifically in the faint end slope, with this slope becoming shallower for redshifts near $z \sim 2$ as compared to the present epoch. There exist proportionately more high equivalent width, high column density ${\MgII}$ absorbers near $z \sim 2$ than at present.
\item The cosmic mass density of ${\MgII}$ absorbing gas, $\Omega_{\hbox{\scriptsize {\MgII}}}$, increases from $\Omega_{\hbox{\scriptsize {\MgII}}} \simeq 0.8\times10^{-8}$ at $z = 0.5$ to $\Omega_{\hbox{\scriptsize {\MgII}}} \simeq 1.3\times10^{-8}$ at $z \sim 2$.
\item The evolution in $d\mathcal{N}\,/dX$ for the highest equivalent width ${\MgII}$ absorbers ($W_r^{\lambda2796} > 1.0$~{\AA}) follows evolutionary trends in the cosmic star formation rate density, which peaks near $z \sim 2$ and falls towards $z \sim 0$. This implies a connection between these very strong systems and star formation driven outflows. Examining other possible factors which could influence the properties of ${\MgII}$ absorbing gas, such as evolution in the intensity of the cosmic ionizing background and changes in cosmic metallicity, we find that the primary effect driving the evolution of the strongest ${\MgII}$ absorbers is the quantity of metal enriched gas expelled through star formation driven feedback.
\item We interpret the evolution of low equivalent width ${\MgII}$ absorbers ($W_r^{\lambda2796} < 0.3$~{\AA}) as a natural consequence of the evolution in the cosmic ionizing background and the metallicity of galaxy halos. These weaker systems potentially originate in the fragmented remains of star formation driven outflows and lower density gas clouds condensing within the host galaxy's cooling radius. They likely represent a lower density, passive component of the CGM, physically distinct from the strongest ${\MgII}$ systems.
\end{enumerate}

\subsection{{\MgII}}
\label{ch8:MgII}

Long song and dance about how MgII is so great and all.

\subsubsection{Strong Absorbers}
\label{ch8:MgIIstrong}

Talk about all the properties of strong absorbers.

\subsubsection{Weak Absorbers}
\label{ch8:MgIIweak}

Talk about all the properties of weak absorbers.

\subsubsection{Kinematics}
\label{ch8:MgIIkinematics}

Talk about kinematic properties of MgII absorbers.

\subsection{{\CIV}}
\label{ch8:CIV}

Even longer song and dance about how CIV is arguably even more interesting and important.

\subsubsection{Strong Absorbers}
\label{ch8:CIVstrong}

Talk about all the properties of strong absorbers.

\subsubsection{Weak Absorbers}
\label{ch8:CIVweak}

Talk about all the properties of weak absorbers.

\subsubsection{Kinematics}
\label{ch8:CIVkinematics}

Talk about kinematics properties of CIV absorbers.

\subsection{Evolution in the Context of Galaxy Evolution}
\label{ch8:evolution}

Evolution.

\subsection{Consequences, and Verification}
\label{ch8:consequences}

Consequences.

\subsection{Future Work}
\label{ch8:futures}

This is where my postdoc stuff would go...IF I HAD ONE.
