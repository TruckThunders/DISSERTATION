\section{\MakeUppercase{Automated Line Detection}}
\label{chp3}

\subsection{Motivating Automated Line Detection}
\label{ch3:motivation}

Large spectral data sets are becomming the norm in both galactic and extragalactic astronomy. In the Milky Way, the spectra of hundreds of thousands of stars may be analyzed by routines which fit templates to find the best agreement. Unfortunately, when dealing with redshifted systems composed of a multitude of components (stars, gas, AGN), the number of free parameters increases significantly and templates/matching techniques become unfeasable. Therefore, we require a more general, but no less robust line finding algorithm to handle large spectral libraries.

\section{Automated Line Detection}
\label{ch3:detection}

We employ a cross-correlation method similar to that of Zhu \& Menard, 2013.

\subsection{Defining the Search Window}
\label{ch3:window}

In any large survey, a well-defined search window is important. In the case of analyzing quasar spectra, we are constrained by several factors: the wavelength coverage of the spectrum, the redshift of the observed quasar, and the confidence with which we can identify individual doublets in regions of low signal-to-noise and regions with atmospheric absorption features (telluric lines).

We limit our wavelength search window for CIV and MgII absorbers to regions redward of the Lyman-alpha emission feature, or the blue limit of the spectrum if the Ly$\alpha$ is not observed, and 3000 km/s blueward of either the corresponding CIV/MgII emission feature, or the red limit of the spectrum if these features are also not observed.

We also do not search in regions of strong telluric absorption, cutting out the following wavelength chunks: [6277 - 6318\AA], [6868 - 6932\AA], [7594 - 7700\AA], and [9300 - 9630\AA].

\subsubsection{The Step-By-Step Recipe for Finding Redshifted Lines}
\label{ch3:steps}

\begin{enumerate}
\item Define Search Window
\item Convert wavelength search window to a redshift range for each ion to search
\item Define redshift resolution based upon the spectrograph resolution
\item Define filter sized based upon FWHM of an unresolved line at the given spectrograph resolution
\item Do the cross-correlation
\begin{itemize}
\item At each redshift, place the filter for each transition (MgII 2796,2803 and CIV 1548,1551)
\item Compute the area between the filter and the observed spectrum - This is the POWER at that redshift. Higher power = stronger absorption.
\item Step forward, doing the same thing. Build up a power spectrum which has the observed power for each transition at a given redshift.
\end{itemize}
\item Normalize the power spectrum
\begin{itemize}
\item The raw power spectrum's amplitude is tied directly to the wavelength range (or redshift range depending on how you want to think about it at this point), as we are taking the area beneath a variable-length object (in this case, the spectrum or filter). We wish to analyze a power spectrum which ranges between 0 and 1 for any spectrum analyzed. This means we want to subtract the baseline value (the result of the correlation if there is zero absorption detected) from the raw power spectrum, and then divide by the difference between the minimum value (no absorption) and the maximum value (a fully saturated absorption line whose width is equal to or greater than the width of the filter used).
\item
\end{itemize}
\end{enumerate}
